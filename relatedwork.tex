\chapter{Related Work}
\label{cha:relatedwork}
\paragraph{Separating Knowledge from Computation: An FO(\textperiodcentered) Knowledge Base System and its Model Expansion Inference \cite{decat14}}
Binnen het domein van declaratief programmeren, ziet men vaak de trend dat zo'n systeem een logische taal implementeert samen met een specifieke vorm van inferentie. Recent heeft men het paradigma Knowledge Base System of KBS voorgesteld met het idee dat kennis over een probleem niet gekoppeld hoeft te zijn aan \'{e}\'{e}n bepaalde vorm van inferentie. In plaats daarvan stelt men voor de kennis apart uit te drukken in een declaratieve taal waarop men dan verschillende vormen van inferentie kan toepassen. In deze paper wordt zo een systeem voorgesteld namelijk IDP. Het laat toe kennis over een bepaald probleem neer te schrijven in FO(\textperiodcentered), oftewel eerste orde logica met een aantal uitbreidingen zoals types, parti\"{e}le functies, aggregaten en inductieve definities. Om mezelf bekend te maken met IDP en het paradigma van kennis representatie, is dit de paper bij uitstek. Het beschrijft het oorspronkelijke doel van IDP en motiveert de keuzes die zijn gemaakt bij het ontwerpen ervan. De paper begint met een korte samenvatting van eerste orde logica, aggregaten en inductieve regels en vervolgens wordt uitgelegd hoe men deze regels uitdrukt in de syntax van het IDP systeem. De verschillende inferentie technieken die ondersteund zijn zoals model expansion, optimization, deduction, model checking enzovoort worden toegelicht. Vervolgens geeft de paper een blik achter de schermen en beschrijft het de moeilijkheden van inferentie en de oplossingen die hiervoor voorzien zijn. Dit is een intro voor iedereen die zich wil verdiepen in IDP, het geeft je de wat, waarom en hoe die je nodig hebt om eraan te beginnen. De syntax in de paper is echter ondertussen niet meer up-to-date. 

\paragraph{Interactive Configurations and the KB-Paradigm \cite{vanhertum}}
Deze paper gaat na of interactieve configuratie problemen kunnen opgelost worden m.b.v. declaratieve methodes en meer specifiek een KR-systeem. Algemeen bekend is dat dit soort problemen enorm complex zijn wil men ze programmeren in een imperatieve programmeertaal \cite{gelle96}. De reden hiervoor is dat regels elkaar be\"{i}nvloeden en een simpele wijziging van ook maar \'{e}\'{e}n regel ervoor kan zorgen dat andere regels niet meer kloppen. De regels zijn ook vaak lang, onoverzichtelijk en slecht leesbaar, problemen die volgens de paper op te lossen zijn met IDP. Gebruik makend van FO(\textperiodcentered) en de syntax van het IDP systeem gaat men proberen de constraints van zo 'n probleem te beschrijven en er vervolgens over redeneren door er verschillende vormen van inferentie op toe te passen. Specifiek wordt er gekeken naar de complexiteit van een aantal verschillende vormen van inferentie die IDP aanbiedt. Geconcludeerd wordt dat ondanks de limitaties van een KR-systeem (grote structuren kunnen tijd van het berekenen opblazen) stelt men toch vast dat de aanpak voldoende goede resultaten boekt. Men kan dit zien als een proof of concept waarbij een echt configuratie probleem uit de bedrijfswereld gebruikt wordt als voorbeeld. Een ISP selecteren is in essentie een interactief configuratie probleem, dus de resultaten die voortvloeien uit deze paper kunnen als leidraad gezien worden voor deze thesis.

\paragraph{A Logical Framework for Configuration Software \cite{vlaeminck09}}
In deze paper neemt de auteur opnieuw de klasse van configuratie problemen onder de loep. In deze verzameling van problemen staat software de gebruiker bij in het maken van bepaalde keuzes volgens een aantal regels. Het schrijven van software hiervoor met imperatieve programmeertalen is vaak een lastige opgave, het is een na\"{i}eve manier van opstellen waarbij elke keuze een if-else regel vereist en de kleinste verandering ervoor kan zorgen dat de regels niet meer kloppen. Dit is voor programmeurs vaak een nachtmerrie als de regels van een systeem geregeld wijzigen (bv. Tax on web). De auteur stelt een declaratieve aanpak voor en specifiek met het oog gericht op het model van kennis representatie. Het doel is om bestaande problemen effici\"{e}nter te kunnen oplossen. Als voorbeeld heeft men gekozen voor het probleem waar een student wordt gevraagd zijn opleiding samen te stellen volgens de regels van de universiteit. De regels kunnen gemakkelijk beschreven worden in IDP, dat gebruik maakt van een uitgebreide versie van FO(\textperiodcentered). De expressiviteit ervan wordt gezien als een groot pluspunt en maakt het heel gemakkelijk voor de programmeur om regels elegant en leesbaar te beschrijven. Er worden verscheidene vormen van inferentie toegepast in de paper waaronder model expansion, propagation, model checking en conflict explanation. Opvallend is dat al deze technieken werken met dezelfde domein logica, \'{e}\'{e}n van de kern idee\"{e}n en grote voordelen van het KR-systeem. Er wordt beschreven hoe elke techniek te werk gaat en uiteindelijk toont de paper aan dat voor het gekozen probleem de inferentie voldoende effici\"{e}nt en in polynomiale tijd verloopt. Hierbij komt ook nog dat de elegantie van een declaratieve aanpak, de grote expressiviteit van FO(\textperiodcentered) en de effici\"{e}ntie van een kennis representatie systeem vele voordelen bieden in tegenstelling tot imperatieve technieken. De technieken die de paper beschrijft voor het oplossen van het gekozen configuratie probleem kunnen als basis dienen voor mijn verder werk en zijn uitermate geschikt om op verder te bouwen. De bedoeling is om technieken toe te voegen zoals minimization en de theorie verder uit te breiden met kennis over het lessenrooster. 

\paragraph{Predicate Logic as a Modelling Language: The IDP System \cite{bogaerts14}}
Recente doorbraken in automated reasoning technieken zoals SAT-solving and constraint programming zijn de aanleiding tot een groeiende interesse in logica als programmeertaal. Ondertussen zijn er al verscheidene declaratieve programmeertalen verschenen. Hier stelt men IDP voor als kennis representatie systeem, waar domein logica volledig onafhankelijk is van de inferentie die men erop wil toepassen. Imperatieve talen laten toe een specifiek algoritme te schrijven dat hoge prestaties kan leveren. Hier tegenover staat dat kennis over het probleem verweven zit in de code, wat onderhoud enorm moeilijk of soms zelfs onmogelijk maakt. Vaak kunnen ze ook maar \'{e}\'{e}n taak uitvoeren, in tegenstelling tot IDP waar verschillende inferentie taken dezelfde domein logica gebruiken. De syntax in IDP is makkelijk te begrijpen en leunt dicht aan bij natuurlijke taal, iets wat bij imperatieve programmeertalen vaak niet het geval is. En tenslotte hoeft de programmeur in IDP zich geen zorgen te maken over hoe een probleem moet opgelost worden. Hij moet enkel de domein logica beschrijven en IDP doet de rest. 
Er zijn ook limitaties, imperatieve oplossingen zijn vaak sneller dan een algemene declaratieve aanpak. Toch worden de prestaties van IDP zeker voldoende geacht. De rekentijd is voldoende kort dat het kan gebruikt worden in een interactieve omgeving waarbij de gebruiker in minder dan enkele seconden een antwoord verwacht. Dit is een van de belangrijkste resultaten van de paper aangezien het aantoont dat IDP wel degelijk in staat is dit en andere gelijkaardige problemen met voldoende effici\"{e}ntie op te lossen. Naast de motivatie van de paper worden alle belangrijke aspecten van IDP uitgelegd. Zo wordt beschreven hoe de FO(\textperiodcentered) in mekaar zit, waarna de paper toont hoe men dit in de syntax van IDP zelfs kan schrijven. Er wordt ook een motivatie gegeven waarom men de keuze van FO(\textperiodcentered) heeft gemaakt, belangrijk hier zijn vooral de expressiviteit, begrijpbaar-heid en herbruikbaarheid. Ook wordt uitgelegd hoe IDP under the hood werkt met technieken zoals ground-and-bound, symmetry exploitation en functional dependency detection.

\paragraph{Consistency Restoration and Explanations in Dynamic CSP's: Application to Configuration \cite{amilhastre01}}
In de klasse van interactieve configuratie problemen is het belangrijk dat de technieken snel en effici\"{e}nt werken. Bepaalde functionaliteiten zoals het voorzien van uitleg bij een foute keuze van de gebruiker en herstellen van consistentie zijn echter nog altijd moeilijk op te lossen en nemen teveel tijd in beslag om ze interactief te kunnen gebruiken. In deze paper proberen ze een oplossing te bieden voor deze problemen. Het voorstel is om een automaat te bouwen die alle verschillende mogelijke selecties voorstelt. De voorwaarde is wel dat deze automaat vooraf off-line wordt opgesteld, dit hoeft maar \'{e}\'{e}n maal te gebeuren. Gebruik makend van deze automaat toont de paper aan dat de voorheen vermelde technieken op een effici\"{e}nte en snelle manier kunnen berekend worden. Om hun worden kracht bij te zetten hebben ze een configuratie probleem van Renault voor het configureren van een auto gebruikt om hun model te testen. De resultaten ervan zijn trouwens zeer indrukwekkend. IDP voorziet momenteel wel een techniek voor het opsporen van conflicten. Maar de technieken voorgesteld in de paper zijn veelbelovend en het is zeker interessant om eens te kijken of we deze toch niet in IDP kunnen integreren. 

\paragraph{Lowering the learning curve for declarative programming: a Python API for the IDP system \cite{vennekens15}}
Nog te vaak zijn programmeurs terughoudend als het aankomt op declaratief programmeren vanwege de leercurve die ermee gepaard gaat. Het doel van de auteur is om de moeilijkheidsgraad te verlagen in de hoop dat het de stap naar declaratieve systemen vergemakkelijkt. De auteur probeert dit te bereiken door een IDP KR-systeem te integreren in een Python API. Python is \'{e}\'{e}n van de meest gebruikte programmeertalen, en dankzij de API kan iedereen vertrouwd met Python gebruik maken van het het IDP KR-systeem zonder de syntax te moeten leren. 
Doch zijn declaratieve systemen inherent verschillend van imperatieve systemen, en iedereen die geen opleiding omtrent declaratieve systemen heeft genoten zal dit dan toch nog moeten inhalen. Bij studenten Computerwetenschappen is dit echter wel het geval aangezien het standaard onderdeel is van de opleiding. En de auteur hoopt met de API dat studenten sneller de stap zullen maken naar IDP, door ze de mogelijkheid te geven om dit te doen in de vaak vertrouwde Python omgeving.
De API is zeker een goede overbrugging, maar ik ben ondertussen bekend met de syntax van IDP. En de effici\"{e}ntie van de syntax en de goede leesbaarheid ervan hebben mij toch doen kiezen om geen gebruik te maken van de API. 

\paragraph{Generating Corrective Explanations for Interactive Constraint Satisfaction \cite{ocallaghan05}}
In het veld van interactieve configuratie problemen is het zo dat als de gebruiker een ongeldige combinatie van variabelen selecteert we op zoek gaan naar de oorzaak hiervan. Anders gezegd gaan we een verklaring zoeken voor het probleem. Dit is natuurlijk handig voor de gebruiker om te achterhalen welke selectie hij moet aanpassen om het conflict op te lossen. Maar nog handiger zou zijn dat niet enkel de oorzaak wordt getoond, maar ook een (of meerdere) mogelijke oplossing(en), m.a.w. een selectie die zo veel mogelijk dezelfde is als die van de gebruiker, maar die wel tot een mogelijk juiste oplossing kan leiden. De paper stelt CORRECTIVEEXP voor, een systematisch algoritme voor het zoeken van \emph{minimal corrective explanations}. Een vergelijkende studie met reeds bestaande technieken op verscheidene grote problemen toont aan dat het algoritme zeer goede resultaten boekt. Dit is essentieel aangezien dat interactieve configuratie problemen een snelle reactietijd vereisen.

\chapter{Related Work}
\label{cha:relatedwork}

Het IDP systeem is reeds beschreven in verscheidene papers die de motivatie, opbouw en werking gedetailleerd uitleggen. Aan de basis liggen vooral de werken van \citep{de2014predicate} en \citep{de2014separating}. Beide papers vormen een basis en een introductie voor iedereen die zich wenst te verdiepen in IDP. Ze beschrijven de onderdelen van FO(\textperiodcentered) en hoe deze kunnen gebruikt worden om de regels van een probleem uit te drukken en hoe deze regels tenslotte gebruikt kunnen worden om meerdere vormen van inferentie te doen. 

\paragraph{Consistency Restoration and Explanations in Dynamic CSP's: Application to Configuration \cite{amilhastre2002consistency}}
In de klasse van interactieve configuratieproblemen is het belangrijk dat de reactietijd van de inferentie technieken laag ligt. Het probleem is echter dat IC problemen net zoals CSP's enorm complex zijn, wat maakt de sommige inferentie technieken intractable (onhandelbaar) zijn in het slechtste geval. In de paper lossen ze dit probleem op door een automaat te bouwen, deze is een compacte weergave van de oplossingsverzameling voor het probleem in kwestie. De voorwaarde is wel dat deze automaat vooraf in een \textit{off-line} compilatiefase wordt opgesteld, dit hoeft maar \'{e}\'{e}n maal te gebeuren. Gebruik makend van deze automaat toont de paper aan dat bepaalde technieken op een effici\"{e}nte en snelle manier kunnen berekend worden. Om hun woorden kracht bij te zetten hebben ze een configuratie probleem van Renault voor het configureren van auto's gebruikt om hun model te testen. De resultaten ervan zijn trouwens zeer indrukwekkend. 

IDP voorziet momenteel wel een techniek voor het opsporen van conflicten, maar kan geen minimale oplossingen berekenen om niet-satisfieerbaarheid op te lossen. De automaat samen met de technieken uit de paper kunnen dit wel, dit bovenop de voordelen van het gebruik van de automaat. Daarom heb ik gekozen voor hier verder op in te gaan en te onderzoeken of deze techniek in combinatie met IDP vruchten kan afwerpen.

\paragraph{Lowering the learning curve for declarative programming: a Python API for the IDP system \cite{vennekens2015lowering}}
Nog te vaak zijn programmeurs terughoudend als het aankomt op declaratief programmeren omwille van de leercurve die ermee gepaard gaat. Het doel van de auteur is om de moeilijkheidsgraad te verlagen in de hoop dat het de stap naar declaratieve systemen vergemakkelijkt. De auteur probeert dit te bereiken door een IDP KR-systeem te integreren in een Python API. Python is \'{e}\'{e}n van de meest gebruikte programmeertalen, en dankzij de API kan iedereen vertrouwd met Python gebruik maken van het het IDP KR-systeem zonder de syntax te moeten leren. Doch zijn declaratieve systemen inherent verschillend van imperatieve systemen, en iedereen die geen opleiding omtrent declaratieve systemen heeft genoten zal dit dan toch nog moeten inhalen. Bij studenten Computerwetenschappen is dit echter wel het geval aangezien het standaard onderdeel is van de opleiding. En de auteur hoopt met de API dat studenten sneller de stap zullen maken naar IDP, door ze de mogelijkheid te geven om dit te doen in de vaak vertrouwde Python omgeving.

\paragraph{Generating Corrective Explanations for Interactive Constraint Satisfaction \cite{o2005generating}}
In het veld van interactieve configuratieproblemen is het zo dat als de gebruiker een ongeldige combinatie van variabelen selecteert het systeem op zoek moet gaan naar de oorzaak hiervan. Anders gezegd gaat het een verklaring zoeken voor het probleem. Dit is natuurlijk handig voor de gebruiker om te achterhalen welke selectie hij moet aanpassen om het conflict op te lossen. Maar nog handiger zou zijn dat niet enkel de oorzaak wordt getoond, maar ook een (of meerdere) mogelijke oplossing(en). Oplossingen tot nu toe werden altijd gezien als de minimale set van variabelen waarvoor de geselecteerde waarde uit het domein ongedaan gemaakt dient te worden, zodanig dat de resterende selectie terug uitgebreid kan worden tot een geldig model. Hier heeft oplossing een geheel andere betekenis. Een oplossing is nog altijd een minimale set van variabelen, maar in plaats van enkel te zeggen dat de keuzes voor deze variabelen ongedaan moet worden gemaakt is er een waarde uit het domein gegeven van de variabelen waarvoor geldt dat ze deel uitmaakt van een geldige selectie. De paper stelt CORRECTIVEEXP voor, een systematisch algoritme voor het zoeken van \emph{minimal corrective explanations}. Een vergelijkende studie met reeds bestaande technieken op verscheidene grote problemen toont aan dat het algoritme zeer goede resultaten boekt. Dit is essentieel aangezien dat interactieve configuratie problemen een snelle reactietijd vereisen.

De inferentietaak die de auteur in zijn paper voorstelt maakt geen deel uit van IDP. Het is een heel interessant concept en zou in de toekomst mogelijk een handige toevoeging kunnen zijn aan het IDP systeem. In mijn zoektocht naar mogelijke technieken heb ik overwogen te kiezen voor deze techniek, maar heb uiteindelijk besloten om dit niet te doen en in plaats daarvan heb ik gekozen voor de implementatie volgens \citep{amilhastre2002consistency}.

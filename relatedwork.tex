\chapter{Related Work}
\label{cha:relatedwork}

\paragraph{Predicate Logic as a Modelling Language: The IDP System \cite{de2014predicate}}
Recente doorbraken in automated reasoning technieken zoals SAT-solving and constraint programming zijn de aanleiding tot een groeiende interesse in logica als programmeertaal. Ondertussen zijn er al verscheidene declaratieve programmeertalen verschenen. Hier stelt men IDP voor als kennis representatie systeem, waar domein logica volledig onafhankelijk is van de inferentie die men erop wil toepassen. Imperatieve talen laten toe een specifiek algoritme te schrijven dat hoge prestaties kan leveren. Hier tegenover staat dat kennis over het probleem verweven zit in de code, wat onderhoud enorm moeilijk of soms zelfs onmogelijk maakt. Vaak kunnen ze ook maar \'{e}\'{e}n taak uitvoeren, in tegenstelling tot IDP waar verschillende inferentie taken dezelfde domein logica gebruiken. De syntax in IDP is makkelijk te begrijpen en leunt dicht aan bij natuurlijke taal, iets wat bij imperatieve programmeertalen vaak niet het geval is. En tenslotte hoeft de programmeur in IDP zich geen zorgen te maken over hoe een probleem moet opgelost worden. Hij moet enkel de domein logica beschrijven en IDP doet de rest. Er zijn ook limitaties, imperatieve oplossingen zijn vaak sneller dan een algemene declaratieve aanpak. Toch worden de prestaties van IDP zeker voldoende geacht. De rekentijd is voldoende kort dat het kan gebruikt worden in een interactieve omgeving waarbij de gebruiker in minder dan enkele seconden een antwoord verwacht. Dit is een van de belangrijkste resultaten van de paper aangezien het aantoont dat IDP wel degelijk in staat is dit en andere gelijkaardige problemen met voldoende effici\"{e}ntie op te lossen. Naast de motivatie van de paper worden alle belangrijke aspecten van IDP uitgelegd. Zo wordt beschreven hoe de FO(\textperiodcentered) in mekaar zit, waarna de paper toont hoe men dit in de syntax van IDP zelfs kan schrijven. Er wordt ook een motivatie gegeven waarom men de keuze van FO(\textperiodcentered) heeft gemaakt, belangrijk hier zijn vooral de expressiviteit, begrijpbaar-heid en herbruikbaarheid. Ook wordt uitgelegd hoe IDP under the hood werkt met technieken zoals ground-and-bound, symmetry exploitation en functional dependency detection. Het werk biedt een overzicht aan van de werking en de mogelijkheden van IDP, kortom het is een goede introductie om te leren werken met het systeem.

\paragraph{Separating Knowledge from Computation: An FO(\textperiodcentered) Knowledge Base System and its Model Expansion Inference \cite{de2014separating}}
Binnen het domein van declaratief programmeren bestaan er ondertussen vele systemen die het mogelijk maken kennis uit te drukken in een logische taal en er vervolgens inferentie op te te passen. Wat opvalt is dat elk systeem vaak gebonden is aan \'{e}\'{e}n enkele vorm van inferentie. Recent heeft men het paradigma Knowledge Base System of KBS voorgesteld met het idee dat kennis over een probleem niet gekoppeld hoeft te zijn aan \'{e}\'{e}n bepaalde vorm van inferentie. De auteur stelt dat door de kennis uit te drukken in een aparte theorie het mogelijk is om meerdere vormen van inferentie toe te passen. In deze paper wordt zo een systeem voorgesteld namelijk IDP. Het laat toe kennis over een bepaald probleem neer te schrijven in FO(\textperiodcentered), oftewel eerste orde logica met een aantal uitbreidingen zoals types, parti\"{e}le functies, aggregaten en inductieve definities. FO(\textperiodcentered) wordt gezien als een rijke taal die dankzij deze uitbreidingen een sterke uitdrukkingskracht heeft. Daarbij is de syntax intu\"{i}tief en gemakkelijk te begrijpen voor de programmeur, zo stelt de paper. Naast de voorstelling van de taal wordt een van de belangrijkste inferentietaken besproken namelijk minimizatie of anders gezegd het genereren van een optimaal model volgens een bepaalde parameter. Er wordt uitgelegd hoe IDP dit aanpakt, de problemen die deze taak met zich meebrengt en hoe IDP deze tracht op te lossen. 

Aangezien dit onderzoek grotendeels bedoeld is om de prestaties van het IDP systeem te testen, is het essentieel dit ik bekend ben met de werking van het systeem. Het werk van De Cat is een z\'{e}\'{e}r goede paper om mee te beginnen. Het beschrijft de bouwstenen van FO(\textperiodcentered) en hoe ze te gebruiken om een theorie te kunnen bouwen. Ondanks dat niet alle inferentietaken worden besproken en de verdieping in de werking van de solver niet meteen van belang is voor mijn thesis, heb ik er toch veel praktische zaken uit kunnen meenemen. Een belangrijke opmerking is wel dat de syntax die de paper gebruikt ondertussen verouderd is. 

\paragraph{The KB Paradigm and its Application to Interactive Configuration \cite{van2016kb}}
De bedoeling van deze paper is de voordelen en de kracht tonen van het KR-systeem IDP bij het oplossen van interactieve configuratieproblemen. Het werk is gebaseerd op dat van \citep{de2014separating} en \citep{denecker2008building}. De gedachtegang is dezelfde, namelijk dat kennis niet samen hoort te hangen met een bepaalde vorm van inferentie. In samenwerking met Adaptive Planet (een consultancy bedrijf) trachten de auteurs een omvangrijk configuratieprobleem van het bedrijf te beschrijven in IDP en vervolgens op te lossen met behulp van de inferentie technieken. De beoordeling van hun resultaten gebeurt op basis van negen criteria volgens \citep{felfernig2001intelligent}. Dit kan gezien worden als een proof of concept waarbij een echt configuratieprobleem uit de bedrijfswereld gebruikt wordt als voorbeeld. 

Een ISP selecteren is in essentie ook een interactief configuratieprobleem, een groot deel van dit onderzoek is opnieuw om de mogelijkheden van IDP te testen in een proof of concept. De opstelling voor mijn werk is z\'{e}\'{e}r gelijkaardig aan dat van \citep{van2016kb}. 

\paragraph{A Logical Framework for Configuration Software \cite{vlaeminck2009logical}}
In deze paper neemt de auteur opnieuw de klasse van interactieve configuratieproblemen onder de loep. In deze verzameling van problemen staat software de gebruiker bij in het maken van bepaalde keuzes volgens een aantal regels. Het schrijven van software hiervoor met imperatieve programmeertalen is vaak een lastige opgave, het is een na\"{i}eve manier van opstellen waarbij elke keuze een if-else regel vereist en de kleinste verandering ervoor kan zorgen dat de regels niet meer kloppen. Dit is voor programmeurs vaak een nachtmerrie als de regels van een systeem geregeld wijzigen (bv. Tax on web). De auteur stelt een declaratieve aanpak voor en specifiek met het oog gericht op het model van kennis representatie. Het doel is om bestaande problemen effici\"{e}nter te kunnen oplossen. Als voorbeeld heeft men gekozen voor het probleem waar een student wordt gevraagd zijn opleiding samen te stellen volgens de regels van de universiteit. De regels kunnen gemakkelijk beschreven worden in IDP, dat gebruik maakt van een uitgebreide versie van FO(\textperiodcentered). De expressiviteit ervan wordt gezien als een groot pluspunt en maakt het heel gemakkelijk voor de programmeur om regels elegant en leesbaar te beschrijven. Er worden verscheidene vormen van inferentie toegepast in de paper waaronder model expansion, propagation, model checking en conflict explanation. Opvallend is dat al deze technieken werken met dezelfde domein logica, \'{e}\'{e}n van de kern idee\"{e}n en grote voordelen van het KR-systeem. Er wordt beschreven hoe elke techniek te werk gaat en uiteindelijk toont de paper aan dat voor het gekozen probleem de inferentie voldoende effici\"{e}nt verloopt. Hierbij komt ook nog dat de elegantie van een declaratieve aanpak, de grote expressiviteit van FO(\textperiodcentered) en de effici\"{e}ntie van een kennis representatie systeem vele voordelen bieden in tegenstelling tot imperatieve technieken. 

De technieken die de paper beschrijft voor het oplossen van het gekozen configuratie probleem kunnen als basis dienen voor mijn verder werk en zijn uitermate geschikt om op verder te bouwen. 

\paragraph{Consistency Restoration and Explanations in Dynamic CSP's: Application to Configuration \cite{amilhastre2002consistency}}
In de klasse van interactieve configuratieproblemen is het belangrijk dat de reactietijd van de inferentie technieken laag ligt. Het probleem is echter dat IC problemen net zoals CSP's enorm complex zijn, wat maakt de sommige inferentie technieken intractable (onhandelbaar) zijn in het slechtste geval. In de paper lossen ze dit probleem op door een automaat te bouwen, deze is een compacte weergave van de oplossingsverzameling voor het probleem in kwestie. De voorwaarde is wel dat deze automaat vooraf in een \textit{off-line} compilatiefase wordt opgesteld, dit hoeft maar \'{e}\'{e}n maal te gebeuren. Gebruik makend van deze automaat toont de paper aan dat bepaalde technieken op een effici\"{e}nte en snelle manier kunnen berekend worden. Om hun woorden kracht bij te zetten hebben ze een configuratie probleem van Renault voor het configureren van auto's gebruikt om hun model te testen. De resultaten ervan zijn trouwens zeer indrukwekkend. 

IDP voorziet momenteel wel een techniek voor het opsporen van conflicten, maar kan geen minimale oplossingen berekenen om niet-satisfieerbaarheid op te lossen. De automaat samen met de technieken uit de paper kunnen dit wel, dit bovenop de voordelen van het gebruik van de automaat. Daarom heb ik gekozen voor hier verder op in te gaan en te onderzoeken of deze techniek in combinatie met IDP vruchten kan afwerpen.

\paragraph{Lowering the learning curve for declarative programming: a Python API for the IDP system \cite{vennekens2015lowering}}
Nog te vaak zijn programmeurs terughoudend als het aankomt op declaratief programmeren omwille van de leercurve die ermee gepaard gaat. Het doel van de auteur is om de moeilijkheidsgraad te verlagen in de hoop dat het de stap naar declaratieve systemen vergemakkelijkt. De auteur probeert dit te bereiken door een IDP KR-systeem te integreren in een Python API. Python is \'{e}\'{e}n van de meest gebruikte programmeertalen, en dankzij de API kan iedereen vertrouwd met Python gebruik maken van het het IDP KR-systeem zonder de syntax te moeten leren. 
Doch zijn declaratieve systemen inherent verschillend van imperatieve systemen, en iedereen die geen opleiding omtrent declaratieve systemen heeft genoten zal dit dan toch nog moeten inhalen. Bij studenten Computerwetenschappen is dit echter wel het geval aangezien het standaard onderdeel is van de opleiding. En de auteur hoopt met de API dat studenten sneller de stap zullen maken naar IDP, door ze de mogelijkheid te geven om dit te doen in de vaak vertrouwde Python omgeving.

De API is zeker een goede overbrugging, maar ik ben ondertussen bekend met de syntax van IDP. En de effici\"{e}ntie van de syntax en de goede leesbaarheid ervan hebben mij toch doen kiezen om geen gebruik te maken van de API. 

\paragraph{Generating Corrective Explanations for Interactive Constraint Satisfaction \cite{o2005generating}}
In het veld van interactieve configuratieproblemen is het zo dat als de gebruiker een ongeldige combinatie van variabelen selecteert het systeem op zoek moet gaan naar de oorzaak hiervan. Anders gezegd gaat het een verklaring zoeken voor het probleem. Dit is natuurlijk handig voor de gebruiker om te achterhalen welke selectie hij moet aanpassen om het conflict op te lossen. Maar nog handiger zou zijn dat niet enkel de oorzaak wordt getoond, maar ook een (of meerdere) mogelijke oplossing(en). Oplossingen tot nu toe werden altijd gezien als de minimale set van variabelen waarvoor de geselecteerde waarde uit het domein ongedaan gemaakt dient te worden, zodanig dat de resterende selectie terug uitgebreid kan worden tot een geldig model. Hier heeft oplossing een geheel andere betekenis. Een oplossing is nog altijd een minimale set van variabelen, maar in plaats van enkel te zeggen dat de keuzes voor deze variabelen ongedaan moet worden gemaakt is er een waarde uit het domein gegeven van de variabelen waarvoor geldt dat ze deel uitmaakt van een geldige selectie. De paper stelt CORRECTIVEEXP voor, een systematisch algoritme voor het zoeken van \emph{minimal corrective explanations}. Een vergelijkende studie met reeds bestaande technieken op verscheidene grote problemen toont aan dat het algoritme zeer goede resultaten boekt. Dit is essentieel aangezien dat interactieve configuratie problemen een snelle reactietijd vereisen.

De inferentietaak die de auteur in zijn paper voorstelt maakt geen deel uit van IDP. Het is een heel interessant concept en zou in de toekomst mogelijk een handige toevoeging kunnen zijn aan het IDP systeem. In mijn zoektocht naar mogelijke technieken heb ik overwogen te kiezen voor deze techniek, maar heb uiteindelijk besloten om dit niet te doen en in plaats daarvan heb ik gekozen voor de implementatie volgens \citep{amilhastre2002consistency}.

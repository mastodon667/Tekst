\chapter{Related Work}
\label{cha:relatedwork}

Het IDP systeem is reeds beschreven in verscheidene papers die de motivatie, opbouw en werking gedetailleerd uitleggen. Aan de basis liggen vooral de werken van \citep{de2014predicate} en \citep{de2014separating}. Beide papers vormen een basis en een introductie voor iedereen die zich wenst te verdiepen in IDP. Ze beschrijven de onderdelen van FO(\textperiodcentered) en hoe deze kunnen gebruikt worden om de regels van een probleem uit te drukken en hoe deze regels tenslotte gebruikt kunnen worden om meerdere vormen van inferentie te doen. 

\paragraph{Lowering the learning curve for declarative programming: a Python API for the IDP system \cite{vennekens2015lowering}}
Nog te vaak zijn programmeurs terughoudend als het aankomt op declaratief programmeren omwille van de leercurve die ermee gepaard gaat. Het doel van de auteur is om de moeilijkheidsgraad te verlagen in de hoop dat het de stap naar declaratieve systemen vergemakkelijkt. De auteur probeert dit te bereiken door een IDP KB-systeem te integreren in een Python API. Python is \'{e}\'{e}n van de meest gebruikte programmeertalen, en dankzij de API kan iedereen vertrouwd met Python gebruik maken van het het IDP KB-systeem zonder de syntax te moeten leren. Doch zijn declaratieve systemen inherent verschillend van imperatieve systemen, en iedereen die geen opleiding omtrent declaratieve systemen heeft genoten zal dit dan toch nog moeten inhalen. Bij studenten Computerwetenschappen is dit echter wel het geval aangezien het standaard onderdeel is van de opleiding. En de auteur hoopt met de API dat studenten sneller de stap zullen maken naar IDP, door ze de mogelijkheid te geven om dit te doen in de vaak vertrouwde Python omgeving.

Conflict Explanation binnen het domein van interactieve configuratie is een breed concept, en de technieken van IDP of die uit \citep{amilhastre2002consistency} zijn niet de enigste de gepubliceerd zijn. In \citep{o2005generating} wordt het concept van \textit{corrective explanations} voorgesteld. Oplossingen tot nu toe werden altijd gezien als de minimale set van variabelen waarvoor de geselecteerde waarde uit het domein ongedaan gemaakt dient te worden, zodanig dat de resterende selectie terug uitgebreid kan worden tot een geldig model. Een Corrective explanation is redelijk gelijkaardig, het is nog altijd een minimale set van variabelen, maar in plaats van enkel te zeggen dat de keuzes voor deze variabelen ongedaan moet worden gemaakt is er een waarde uit het domein gegeven van de variabelen waarvoor geldt dat ze deel uitmaakt van een geldige selectie. De paper stelt CORRECTIVEEXP voor, een systematisch algoritme voor het zoeken van \emph{minimal corrective explanations}. Een vergelijkende studie met reeds bestaande technieken op verscheidene grote problemen toont aan dat het algoritme zeer goede resultaten boekt in de zin dat het snel oplossingen kan vinden. Dit is uiteraard essentieel aangezien dat interactieve configuratie problemen een snelle reactietijd vereisen.

/*ANDERE STUDIES BESCHRIJVEN*/

\chapter{Related Work}
\label{cha:relatedwork}

Het IDP systeem is reeds beschreven in verscheidene papers die de motivatie, opbouw en werking gedetailleerd uitleggen. Aan de basis liggen vooral de werken van \citep{de2014predicate} en \citep{de2014separating}. Beide papers vormen een basis en een introductie voor iedereen die zich wenst te verdiepen in IDP. Ze beschrijven de onderdelen van FO(\textperiodcentered) en hoe deze kunnen gebruikt worden om de regels van een probleem uit te drukken en hoe deze regels tenslotte gebruikt kunnen worden om meerdere vormen van inferentie te doen. 
\par
In een poging om de kloof tussen IDP en imperatieve programmeertalen te dichten, stelt \citep{vennekens2015lowering} in zijn werk de ontwikkeling van een API voor. Deze API die geschreven is in python, maakt het mogelijk om IDP te gebruiken vanuit de vertrouwde omgeving. Python is \'{e}\'{e}n van de meest gebruikte programmeertalen en de auteur hoopt daarmee zoveel mogelijk mensen aan te spreken. De API laat toe om regels te beschrijven met de syntax van python, waardoor een ontwikkelaar zich niet bekend met FO(\textperiodcentered) moet maken. Een gepaste achtergrond in declaratief programmeren is echter wel vereist.
\par
Conflict Explanation binnen het domein van interactieve configuratie is een breed concept, en de technieken van IDP of die uit \citep{amilhastre2002consistency} zijn niet de enigste de gepubliceerd zijn. In \citep{o2005generating} wordt het concept van \textit{corrective explanations} voorgesteld. Oplossingen tot nu toe werden altijd gezien als de minimale set van variabelen waarvoor de geselecteerde waarde uit het domein ongedaan gemaakt dient te worden, zodanig dat de resterende selectie terug uitgebreid kan worden tot een geldig model. Een Corrective explanation is redelijk gelijkaardig, het is nog altijd een minimale set van variabelen, maar in plaats van enkel te zeggen dat de keuzes voor deze variabelen ongedaan moet worden gemaakt is er een waarde uit het domein gegeven van de variabelen waarvoor geldt dat ze deel uitmaakt van een geldige selectie. De paper stelt CORRECTIVEEXP voor, een systematisch algoritme voor het zoeken van \textit{minimal corrective explanations}. Een vergelijkende studie met reeds bestaande technieken op verscheidene grote problemen toont aan dat het algoritme zeer goede resultaten boekt in de zin dat het snel oplossingen kan vinden. Dit is uiteraard essentieel aangezien dat interactieve configuratie problemen een snelle reactietijd vereisen.

/*ANDERE STUDIES BESCHRIJVEN*/

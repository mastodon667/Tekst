\chapter{Implementation}
\label{cha:implementation}

\subsection{Preliminaries}
De kennis over het ISP staat beschreven in een theorie $\mathcal{T}$ over een vocabulair $\Sigma$. Deze vocabulair bestaat uit een set $\Sigma\textsubscript{p}$ van predicaatsymbolen en een set van functiesymbolen $\Sigma\textsubscript{f}$. Een interpretatie $\mathcal{I}$ bestaat uit een $\mathcal{D}$ de verzameling domeinelementen, een mapping van elk functiesymbool f/n naar een functie met ariteit n op $\mathcal{D}$ en een mapping van elk predicaatsymbool P/n naar een relatie R $\subseteq$ $\mathcal{D}\textsuperscript{n}$. Een three-valued parti\"{e}le interpretatie bestaat uit een mapping op drie waarheidswaarden {u,t,f} respectievelijk onbekend, waar en niet waar. Deze is minder precies dan een two-valued interpretatie op {t,f}. 
De waarheidswaarden zijn partieel geordend, u $\leq\textsubscript{p}$ f en u $\leq\textsubscript{p}$ t.

\section{Knowledge base constructie}
Geen opleiding aan de KU Leuven is dezelfde, de opleidingsonderdelen verschillen net zoals de vakgroepen waar ze deel van uitmaken. Voor sommige vakgroepen ben je verplicht alle onderdelen op te nemen, terwijl je voor andere en minimaal (of zelfs maximaal) aantal studiepunten ervan moet opnemen. Kortom de regels binnen \'{e}\'{e}nzelfde opleiding zijn niet zo moeilijk om te beschrijven, maar om dit te veralgemenen en een theorie te cre\"{e}ren die voor alle opleidingen een geldig ISP kan beschrijven dat is een andere paar mouwen. Voor het gemak van de lezer noemen we opleidingsonderdelen vanaf nu vakken. Hieronder staan de verschillende onderdelen van de knowledge base beschreven.

\subsection{Vocabulary}
Om te beginnen zal ik de vocabulair toelichten, beginnend bij de types waarna ik de functies en predicaten zal beschrijven.
\begin{description}
\item [Vak] De naam verklaart zichzelf al, dit type omvat de verzameling van de verschillende vakken in het domein. 
De domeinwaarden zijn niet de naam in natuurlijke taal, maar de unieke vakcode.
\item [VakNaam] Elk vak heeft een naam geschreven in natuurlijke taal oftwel de vaknaam.
\item [VakGroep] Vakgroepen zijn een heel algemeen concept, het omvat elke verzameling van vakken waarvoor bepaalde regels gelden. Per groep gelden er andere regels die uniek zijn voor die groep, maar ook het type waartoe de groep behoort.
\item [VakGroepType] Elke vakgroep maakt deel uit van een bepaald type groep, en per type groep gelden er bepaalde regels bovenop de regels van de individuele groep zelf.
\item [Fase] Opleidingen bestaan uit fases oftewel jaren, de Master Computer Wetenschappen bestaat bijvoorbeeld uit 2 fases. Vakken binnen een opleiding kunnen enkel gevolgd worden tijdens de fase van de opleiding waartoe ze behoren. 
\item [Semester] De K.U. Leuven werkt met een semester systeem, dit wil zeggen dat elke fase (schooljaar) is opgedeeld in 2 semesters. Elk vak kan gevolgd worden in de semester waartoe het behoort. Er bestaan ook jaarvakken waarvoor de werklast verdeeld is tussen beide semesters, een voorbeeld hiervan is de Masterproef.
\item [Studiepunten] Elk vak bevat studiepunten, deze beschrijven de geschatte werklast voor dit vak. Vakgroepen vereisen vaak dat je een bepaalde hoeveelheid een studiepunten opneemt. 
\end{description}

\begin{description}
\item [IsType(VakGroep,VakGroepType)]
\item [Verplicht(Vak,VakGroep)]
\item [InVakGroep(Vak,VakGroep)]
\item [InFase(Vak,Fase)]
\item [Geselecteerd(Vak,Fase)]
\item [NietGeselecteerd(Vak)]
\end{description}

\begin{description}
\item [HeeftNaam(Vak):VakNaam]
\item [MinAantalStudiepunten(VakGroep):Studiepunten]
\item [MaxAantalStudiepunten(VakGroep):Studiepunten]
\item [AantalStudiepunten(Vak):Studiepunten]
\item [GeselecteerdAantalStudiepuntenPerVakGroep(Vakgroep):Studiepunten]
\item [GeselecteerdAantalStudiepuntenPerSemester(VakGroep):Studiepunten]
\item [InSemester(Vak):Semester]
\end{description}

\subsubsection{Vakgroeptypes}
Zo valt op dat we telkens dezelfde \textbf{types} van \emph{vakgroepen} tegenkomen. Deze types die steeds terugkomen in meerdere opleidingen hebben ongeacht de opleiding waar ze in voorkomen dezelfde regels die ermee gepaard gaan. Zo is elke opleiding op zich een vakgroep van het type \emph{opleiding}, het bevat vakken die je verplicht bent te volgen en keuzevakken. Het heeft een minimum (en maximum) aantal studiepunten dat de student verplicht moet opnemen. Een ander belangrijk type vakgroep is de (hoofd)specialisatie, verscheidene opleidingen geven de keuze tussen meerdere specialisaties. Verwacht wordt dat je van minstens 1 zo'n specialisatie alle verplichte vakken opneemt. Naast hoofdspecialisatie bestaat er ook het type verdere specialisatie, waarin de student verplicht wordt een bepaald aantal studiepunten op te nemen aan vakken uit deze of bepaalde andere vakgroepen. Vervolgens is er het type 'Algemeen vormende en onderzoeksondersteunende groep' vakgroep, dit is veruit het moeilijkste type groep om te beschrijven. De student moet opnieuw een bepaald aantal studiepunten aan vakken opnemen uit deze groep. Maar daarnaast gelden er ook specifieke regels voor verscheidene vakken die deel uitmaken ervan. En als laatse is er het type 'Bachelor verbredend pakket', waarin studenten die beginnen aan hun masteropleiding vakken moeten opnemen die ontbraken in hun bacheloropleiding.

Deze types zien we het vaakst voorkomen en kunnen ongeacht de opleiding met dezelfde set van regels beschreven worden. Het is mogelijk dat sommige types van vakgroepen niet telkens voorkomen, en dus de regels ook niet van kracht zijn. In een bacheloropleiding zal bijvoorbeeld nooit een vakgroep voorkomen van het type Bachelor verbrendend pakket. 

\subsection{Theorie van het lessenrooster}



\section{Front-End}
Herinner dat het samenstellen van het ISP een Interactief configuratieprobleem is, en om aan de eigenschappen van zo'n IC probleem te kunnen voldoen moet de reactietijd van de inferentiemethoden snel genoeg zijn. Om dit te testen heb ik een front-end applicatie ontwikkeld die met behulp van een grafische interface een student toelaat een ISP samen te stellen. Deze input wordt vertaald naar FO(\textperiodcentered) en doorgespeeld naar IDP. Vervolgens voert IDP inferentie uit en de resultaten hiervan worden terug doorgespeeld naar de front-end, die het resultaat zal weergeven d.m.v. de GUI. Zo kunnen de prestaties van IDP in een gebruiksvriendelijke omgeving getest worden. 

\subsection{Grafische Gebruikersinterface}



\subsection{Communicatie tussen Front- en Back-end}
Niet alleen zijn is de syntax tussen IDP en Python compleet verschillend, het zijn ook twee verschillende klassen van programmeertalen respectievelijk declaratief en objectgeori\"{e}nteerd. Dus communicatie tussen de twee talen is niet vanzelfsprekend. Een Python API die toelaat IDP te gebruiken is reeds ontwikkeld \citep{vennekens2015lowering}. En hoewel deze API zeker de kloof tussen beide programmeertalen kleiner maakt, schiet ze op sommige vlakken toch te kort. Zo laat het bijvoorbeeld niet toe om aggregaten te te beschrijven. Daarom heb ik gekozen om ze niet te gebruiken en in plaats daarvan zelf de communicatie tussen beide te regelen. 

De input/output van IDP is compleet tekstueel, de front-end moet een volledig tekst-bestand parsen in een formaat dat IDP kan begrijpen en verwerken. De output van IDP is opnieuw een tekst-bestand dat de front-end zal moeten ontcijferen om het resultaat te kunnen verwerken. Zo'n tekst-bestand heeft altijd dezelfde opmaak, het bevat een theorie, vocabulair, structuur, procedure en eventueel termen in geval van minimizatie. De front-end bevat een parser object dat voor elke oproep zo'n tekst bestand zal opbouwen. De theorie, vocabulair en de procedures zijn altijd dezelfde en hoeven slecht \'{e}\'{e}nmalig ingelezen te worden. Enkel de structuur zal bij elke nieuwe actie gedeeltelijk dynamische gegenereerd moeten worden. De predicaten in de vocabulair zijn opgedeeld in twee verzamelingen. De eerste verzameling $\Gamma$ bevat de predicaten die vooraf gegeven zijn en gekend zijn door het systeem. Anders gezegd bestaat er een interpretatie $\mathcal{G}$ voor $\Gamma$. De andere verzameling $\Omega$, zijn de predicaten waar nog geen interpretatie $\mathcal{W}$ voor gegeven is en die door de gebruiker (in dit geval de student) ingevuld zullen moeten worden. Tot deze laatste groep behoren volgende predicaten: Geselecteerd(Vak,Fase) en NietGeinteresseerd(Vak). De bedoeling is om een invulling te zoeken zodat $\mathcal{W}$ $\cup$ $\mathcal{G}$ consistent is met alle regels van de theorie $\mathcal{T}$.


\section{Features}

\subsection{Selectieproces}
Elke keuze die de gebruiker maakt moet zorgvuldig afgehandeld worden om te kunnen geranderen dat de selectie altijd satisfieerbaar is. De voorwaarde is dat als de gebruiker een keuze maakt die ervoor zorgt dat de theorie niet langer satisfieerbaar is, hij hier meteen op de hoogte van wordt gebracht en zodanig dat hij dit kan oplossen alvorens verder te gaan. Er wordt gestart met de minst precieze interpretatie $\mathcal{W}\textsubscript{0}$ om uiteindelijk stapsgewijs tot een two-valued interpretatie $\mathcal{W}\textsubscript{n}$ te komen waarvoor geldt dat $\mathcal{W}\textsubscript{n}$ $\cup$ $\mathcal{G}$ $\models$ $\mathcal{T}$. Hoe deze stappen verlopen staat beschreven in onderstaand algoritme. Hier wordt $\mathcal{W}$ opnieuw opgedeeld in twee verzamelingen $\mathcal{U}$ en $\mathcal{P}$, respectievelijk de keuzes gemaakt door de gebruiker en de propagaties die hieruit volgen. 

\begin{algorithm}
	\SetKwInOut{Input}{Input}
	\SetKwInOut{Output}{Output}
	\underline{function Selectieproces} $(\mathcal{U}\textsubscript{i})$ \;
	\Input{De nieuwe keuze van de gebruiker $\mathcal{U}\textsubscript{i}$}
	\Output{$\mathcal{U}\textsubscript{i}$, nieuwe propagaties $\mathcal{P}\textsubscript{i}$}
	$\mathcal{U}\textsubscript{i} \leftarrow \mathcal{U}\textsubscript{i-1}$ [$\mathcal{N}\textsubscript{i}$]\;
	\eIf{Sat($\mathcal{U}$)}
		{
		$\mathcal{P}\textsubscript{i-1} \leftarrow$ Propagate($\mathcal{U}\textsubscript{i}$)\;
		*magic*\;
		\eIf{$\mathcal{D}$ = $\emptyset$}
			{
			createAction()\;
			}
			{
			
			}
		}
		{
		$\mathcal{V} \leftarrow$ Unsat($\mathcal{U}\textsubscript{i}$)\; 
		}
	\caption{Selectieproces}
\end{algorithm}

\subsection{Propagatie}
Om het ISP samen te stellen moeten volgende, nog onbekende predicaten ingevuld worden door de gebruiker: Geselecteerd(Vak,Fase) en NietGeinteresseerd(Vak). Bij elke stap (nieuwe selectie die de gebruiker maakt) wordt indien die selectie satisfieerbaar is, eventuele propagaties ook geselecteerd. 

\subsection{Model Expansie}
Als de gebruiker een aantal selecties heeft gedaan en voor de rest geen specifieke eisen meer heeft, kan hij vragen aan het systeem om het ISP verder te laten genereren. 

\subsection{Minimizatie}
Ook is het mogelijk om het systeem het beste ISP te laten genereren volgens bepaalde regels. Momenteel is het mogelijk om een ISP te zoeken dat zo weinig mogelijk werklast bevat (zo min mogelijk studiepunten). Of om de werklast zo goed mogelijk te verdelen over de beide semesters. 
Daarnaast is het ook mogelijk om de vakken zo te selecteren dat hun lesmomenten zo weinig mogelijk overlappen. 



\subsection{Ongedaan Maken}
Het ongedaan kunnen maken van selecties is een van de aspecten waar meerdere strategi\"{e}en mogelijk zijn. Belangrijk om weten is dat er een duidelijk onderscheid gemaakt moet worden tussen de keuzes gemaakt door de gebruiker en de eventuele propagaties die hieruit volgen. Deze moeten dan ook voor elke tussenstap keurig bijgehouden worden in wat we vanaf nu een actie zullen noemen. Het is eveneens belangrijk om de originele waarden bij te houden, zodanig dat als men een actie ongedaan maakt de toestand terug kan gezet worden naar hoe ze voorheen was. Het voorgaande is allemaal redelijk vanzelfsprekend, maar de moeilijkheid zit hem in de verschillende mogelijke strategie\"{e}n. 

\paragraph{Strategie 1}
Een mogelijkheid is om simpelweg een actie te zien zoals een overgang van \'{e}\'{e}n consistente toestand naar een nieuwe consistente toestand. Dus elke actie weet hoe de selectie eruit zag voordat ze uitgevoerd was, en hoe de selectie eruit zag na afloop van de actie. Een actie ongedaan maken houdt dan simpelweg in dat je de nieuwe toestand terug vervangt door de oude, een beetje als een rollback operatie. De voorwaarde hier is dat enkel de laatste actie ongedaan gemaakt kan worden, alvorens mijn de voorlaatste ongedaan wil maken. 

\paragraph{Strategie 2}
Dat brengt ons bij het volgende, bij het ongedaan maken van een actie draaien we als het ware de tijd terug en wordt de huidige toestand vervangen door de voorgaande. Dit wil zeggen dat ook de propagaties uit de actie verdwijnen, en misschien wil de gebruiker dit niet. In plaats daarvan willen we bij het ongedaan maken van een operatie de gebruiker laten kiezen of de eventuele propagaties mogen geselecteerd blijven of dat ze ook ongedaan gemaakt mogen worden. Hier komen we het volgende probleem tegen. Stel dat het vak A geselecteerd is en hieruit volgt dat vak B ook gevolgd moet worden. In de volgende actie selecteert de gebruiker dat hij niet ge\"{i}ntereseerd is in A en hieruit volgt dat B niet gevolgd mag worden. De gebruiker komt terug op zijn beslissing en wil de laatste actie ongedaan maken. Dit wil zeggen dat de keuze van vak A van niet ge\"{i}ntereseerd terug gezet wordt naar wel geselecteerd. Hieruit volgt dat de propagatie over vak B niet meer van kracht is, en de gebruiker dus de keuze krijgt om deze ongedaan te maken of niet. Stel dat hij ervoor kiest de propagatie te behouden dan wordt het probleem onsatisfieerbaar. Dit is niet het enigste probleem dat zich voordoet met deze strategie. Iedere actie beschrijft hoe de toestand voor en na de actie. Als de gebruiker een actie ongedaan maakt maar ervoor kiest om de propagaties uit deze actie toch te behouden, dan wil dit dus zeggen dat het resultaat van de voorgaande actie niet meer klopt met de huidige situatie. In feite maken we dus niet heel de actie ongedaan wat het heel moeilijk maakt om een duidelijk overzicht te bieden en consistentie te garanderen. Dat brengt ons bij het volgende punt. We willen actie Ai ongedaan maken, de actie bevat Ui de keuze van de gebruiker en Pi de propagaties die eruit volgden. De gebruiker kiest ervoor om p1 $\in$ Pi te behouden. Ai wordt ongedaan gemaakt, maar nu wordt er een actie A'i toegevoegd waarin U'i ={p1} de nieuwe selectie van de gebruiker is en P'i de eventuele propagaties die hieruit volgen. 

\paragraph{Strategie 3}
Tenslotte is er nog een geheel andere optie, waarbij niet meer uit wordt gegaan van acties die men kan ongedaan maken. Een vak kan geselecteerd zijn in een fase, het kan zijn dat de gebruiker niet ge\"{i}nteresseerd is in een vak of het kan zijn dat de gebruiker nog geen keuze heeft gemaakt. Het idee van iets ongedaan maken bij deze strategie is het volgende: de gebruiker heeft geselecteerd dat hij/zij vak A wil volgen tijdens fase 1 van de opleiding. Later zegt de gebruiker dat hij zij dit niet meer wel, maar zegt verder specifiek niets over het vak. We hebben de volgende verzamelingen: Ci de vakken waar de gebruiker specifiek een keuze over heeft gemaakt, Pi de propagaties die volgen uit Ci en Ui de vakken waarover nog geen uitspraak is gedaan. Stel vak V $\in$ Cn-1 maar de gebruiker maakt de selectie ongedaan en V wordt toegevoegd aan Un. Dus Cn = Cn-1 - V. dan gaan we kijken wat Pn en Un zijn, en vervolgens gaan we zien wat de doorsnede is van Pn-1 en Un, is deze verzameling niet leeg, dan wil dit zeggen dat er propagaties zijn die niet langer bestaan, en de gebruiker moet kiezen of hij/zij ze wil behouden. Zo ja worden ze toegevoegd aan Cn+1 en begint het hele process opnieuw.

\section{Conflict Explanation}
Het is mogelijk dat de gebruiker een verkeerde keuze maakt waardoor de theorie nietmeer satisfieerbaar is. Simpel gezegd heeft de gebruiker een waarde gekozen voor een bepaalde variabele, zodanig dat dit ervoor zorgt dat \'{e}\'{e}n of meerdere regels uit de theorie niet meer waar kunnen worden gemaakt.

\begin{lstlisting}[mathescape]
A $\wedge$ B
$\neg$ A
\end{lstlisting}

In het voorbeeld, zegt de theorie ons dat A en B beide waar moeten zijn. We zeggen dat A niet waar is. Door deze invulling kan de regel nooit waar worden ongeachte de waarde van B. De theorie is dus onsatisfieerbaar. 

Conflict explanation wil zeggen een verklaring zoeken waarom een bepaalde selectie door de gebruiker dit veroorzaakt. 

IDP gebruikt momenteel twee technieken om oorzaken van onsatisfieerbaarheid op te sporen. De unsatstructure spoort de set van variabelen op die het probleem veroorzaken. Vervolgens is er de unsatcore, die zoekt naar de kleinste set van regels uit de theorie die niet waar gemaakt kunnen worden.

\subsubsection{Unsatstructure}
De unsatstructure is een effici\"{e}nte tool om aan de gebruiker te kunnen meedelen, welke van de voormalige selecties problemen veroorzaken. In ons voorbeeld van het ISP kan de gebruiker enkel de keuze maken om een dag te volgen in een bepaalde fase (te kiezen uit de lijst van mogelijke waarden) of kan hij/zij aangeven om een vak niet te volgen. 
Bij een slechte selectie zal de unsatstructure de vakken oplijsten samen met de selectie die de gebruiker ervoor gemaakt heeft, en dit enkel voor de vakken waarvoor de selectie bijdragen tot de onsatisfieerbaarheid. De gebruiker krijgt hier dan de mogelijkheid om voor deze vakken de voorheen gemaakte keuze ongedaan te maken of te veranderen. 

\subsubsection{Reified Constraints}
Wat opvalt is dat de voorgaande techniek enkel een verzameling selecties teruggeeft die bijdragen tot het probleem. Maar verdere uitleg over welke regel(s) uit de theorie niet meer waar gemaakt kunnen worden en waarom wordt niet gegeven. Dus de gebruiker kan niet weten wat er moet veranderen. 
Herinner de unsatcore zoekt achter de kleinste set van regels uit de theorie die niet meer waar gemaakt kunnen worden. En hoewel dit hetgene is dat we willen weten, is er toch nog een probleem. 
De IDP syntax is voor de doorsnee programmeur goed te lezen, maar voor een doorsnee gebruiker die niets van programmeren afweet zal dit ongetwijfeld als chinees overkomen. Neem als voorbeeld onderstaande regel, elke student computerwetenschappen zal vrij snel kunnen achterhalen wat de regel inhoud. Maar personen zonder deze achtergrond zullen dit niet kunnen verstaan.
\begin{lstlisting}[mathescape, caption=IDP Rule Example, frame=topline/bottomline]
$\forall$vg[VakGroep] : IsType(vg,AVO) $\Rightarrow$ GesAantalStupunVakGr(vg) 
= sum{v[Vak], sp[Studiepunten],f[Fase] : InVakGroep(v,vg) 
$\wedge$ Geselecteerd(v,f) $\wedge$ AantalStudiepunten(v)=sp : sp }.
\end{lstlisting}
We proberen dit probleem deels te overbruggen d.m.v. reified constraints. Hierbij koppelen we de waarheidswaarde van een regel (constraint) uit de theorie aan een booleaanse variabele. Nemen we terug onze regel uit voorbeeld 1, daarvoor plaatsen we de Booleaanse variabele C toe gevolgd door een equivalentie. Wat dit wil zeggen is dat de waarheidswaarde voor en na de equivalentie dezelfde moet zijn. Dus als de regel uit voorbeeld 1 niet voldaan is, en dus 'false' is dan moet C ook 'false' zijn. 
\begin{lstlisting}[mathescape]
C $\Leftrightarrow$ A $\wedge$ B
$\neg$ A
\end{lstlisting}
Op deze manier hoeven we enkel na te gaan of C true of false is om te weten te komen of de regel waar is of niet. Hiermee is de gebruiker natuurlijk nog niets wijzer geworden, dus wat is nu net het nut van dit soort constraints? Voor elke regel uit de theorie hebben we een uitleg in natuurlijke taal geschreven die beschrijft wat er mis is als de regel niet consistent is. Daarna moeten we simpelweg voor elke regel controleren wat de waarheidswaarde is van de booleaanse variabele voor de equivalentie. Is deze false, dan is de regel dus inconsistent en krijgt de gebruiker de uitleg te zien van deze regel in natuurlijke taal. 

Hoewel dit een zeer gemakkelijke en generische manier is om aan conflict explanation te doen, zijn de mogelijkheden van deze techniek toch gelimiteerd en niet altijd voldoende toereikend. Ten eerste is de uitleg vaak oppervlakkig en niet specifiek genoeg om duidelijk te kunnen aanwijzen welke selectie juist verantwoordelijk is voor de inconsistentie. /*VOORBEELD PRESENTATIE INVOEGEN*/ En ten tweede laat deze methode enkel toe om inconsistente regels op te sporen. Een regel is inconsistent als ze niet voldaan is gegeven de huidige parti\"{e} selectie. Terwijl een set van regels die nooit samen waar gemaakt kunnen worden gegeven de huidige parti\"{e}le selectie niet-satisfieerbaar zijn. 



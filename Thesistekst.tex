\documentclass[master=elt,masteroption=eg,english]{kulemt}
\setup{title={The ISP Course Selection Puzzle},
  author={Herbert Gorissen},
  promotor={Prof.\ Gerda Janssens},
  assessor={},
  assistant={Matthias van der Hallen}}
% The following \setup may be removed entirely if no filing card is wanted
\setup{filingcard,
  translatedtitle=The ISP Course Selection Puzzle,
  udc=621.3,
  shortabstract={Here comes a very short abstract, containing no more than 500
    words. \LaTeX\ commands can be used here. Blank lines (or the command
    \texttt{\string\pa r}) are not allowed!
    \endgraf}}
% Uncomment the next line for generating the cover page
%\setup{coverpageonly}
% Uncomment the next \setup to generate only the first pages (e.g., if you
% are a Word user.
%\setup{frontpagesonly}

%\begin{figure}
%  \centering
%  \includegraphics{logokul}
%  \caption{The KU~Leuven logo.}
%  \label{fig:logo}
%\end{figure}

% Choose the main text font (e.g., Latin Modern)
\setup{font=lm}

% If you want to include other LaTeX packages, do it here. 

% Finally the hyperref package is used for pdf files.
% This can be commented out for printed versions.
\usepackage[pdfusetitle,colorlinks,plainpages=false]{hyperref}
\usepackage{natbib}
\usepackage{listings}
\bibliographystyle{plainnat}
\setcitestyle{authoryear}

%%%%%%%
% The lipsum package is used to generate random text.
% You never need this in a real master thesis text!
\IfFileExists{lipsum.sty}%
 {\usepackage{lipsum}\setlipsumdefault{11-13}}%
 {\newcommand{\lipsum}[1][11-13]{\par And some text: lipsum ##1.\par}}
%%%%%%%

%\includeonly{chap-n}
\begin{document}

\begin{preface}
  
\end{preface}

\tableofcontents*

\begin{abstract}
Bij de start van een opleiding aan de KU Leuven is iedere toekomstige student verplicht zijn of haar opleiding samen te stellen uit een hele waaier aan opleidingsonderdelen. De selectie van opleidingsonderdelen is echter onderworpen aan een set van regels die allemaal voldaan moeten zijn wil men een geldig individueel studieprogramma (ISP) bekomen. Het ISP maakt deel uit van een specifieke groep van problemen genaamd emph{configuratie problemen}.

Het IDP systeem ontwikkeld aan de KU Leuven laat toe domein specifieke kennis uit te drukken in FO(\textperiodcentered), een uitbreiding op eerste orde logica, ook wel de theorie genoemd. Eens de kennis beschreven is kan men met het bijhorende IDP systeem verscheidene vormen van inferentie toepassen op de theorie. 

IDP als kennis representatie systeem is ideaal voor het beschrijven van en later redeneren over configuratie problemen. In het verleden heeft men er al grote configuratie problemen uit de bedrijfswereld succesvol mee kunnen beschrijven en oplossen. Interessant is om te achterhalen of we dit voor het ISP selectie probleem ook kunnen doen. 

Momenteel zijn er bij het opstellen van het ISP ook een aantal tekortkomingen die we met IDP zouden willen oplossen. Zo wordt het lessenrooster niet mee in rekening gebracht, waardoor studenten niet weten of ze opleidingsonderdelen opnemen waarvan de lessen mogelijk kunnen overlappen. En wat als er een foutieve selectie wordt gemaakt? Momenteel krijgt de student na het bevestigen van diens selectie een foutmelding en deze is vaak nog onduidelijk. IDP voorziet momenteel een aantal functies om de oorzaken van inconsistentie op te sporen at run-time. Handig zou zijn om hiermee korter op de bal te spelen en meteen bij een foutieve selectie uitleg te kunnen geven waarom de voorlopige huidige selectie niet correct is en hoe men dit kan oplossen.

Naast dit alles heeft deze thesis nog een andere grote doelstelling. Het opsporen van oorzaken van inconsistenties ofwel conflict explanation genoemd is een gebied waar momenteel nog veel onderzoek gaande is. Enkele interessante technieken zijn hier al voorgelegd en het is in onze interesse om na te gaan of en hoe we deze kunnen integreren in het IDP systeem.
\end{abstract}

% A list of figures and tables is optional
%\listoffigures
%\listoftables
% If you only have a few figures and tables you can use the following instead
\listoffiguresandtables
% The list of symbols is also optional.
% This list must be created manually, e.g., as follows:
\chapter{List of Abbreviations and Symbols}
\section*{Abbreviations}
\begin{flushleft}
  \renewcommand{\arraystretch}{1.1}
  \begin{tabularx}{\textwidth}{@{}p{12mm}X@{}}
    LoG   & Laplacian-of-Gaussian \\
    MSE   & Mean Square error \\
    PSNR  & Peak Signal-to-Noise ratio \\
  \end{tabularx}
\end{flushleft}
\section*{Symbols}
\begin{flushleft}
  \renewcommand{\arraystretch}{1.1}
  \begin{tabularx}{\textwidth}{@{}p{12mm}X@{}}
    42    & ``The Answer to the Ultimate Question of Life, the Universe,
            and Everything'' according to \\
    $c$   & Speed of light \\
    $E$   & Energy \\
    $m$   & Mass \\
    $\pi$ & The number pi \\
  \end{tabularx}
\end{flushleft}

% Now comes the main text
\mainmatter

\chapter{Introduction}
\label{cha:intro}

\subsection{Constraint Satisfaction Problems}
CSP's zijn problemen waar voor een reeks variabelen een geldige waarde uit het domein dient toegekend te worden. De toekenning echter is gebonden aan een set van regels waaraan de variabelen moeten voldoen. Formeel defini\"{e}ren we een CSP als een triple $\langle \mathcal{X},\mathcal{D},\mathcal{C} \rangle$, met $\mathcal{X}$ de verzameling van variabelen, $\mathcal{D}$ de domeinwaarden voor deze variabelen en $\mathcal{C}$ de constraints over de variabelen. Typisch worden dit soort problemen opgelost d.m.v. zoeken door alle mogelijk combinaties, deze recursieve methode gaat voor elke nieuwe toekenning na of de constraints voor de parti\"{e}le toekenning consistent zijn. Zo ja, dan volgt er een nieuwe recursieve oproep, anders gaat het algoritme backtracken. Het oplossen van CSP's met een eindig domein behoort tot de klasse van NP-Complete problemen en de berekeningen zijn vaak van hoge complexiteit met betrekking tot de grootte van het domein. Om het zoekproces proberen te versnellen bestaan er variaties op backtracking zoals backmarking en backjumping. De eerste zorgt voor een effici\"{e}ntere manier om consistentie na te gaan en backjumping is een betere manier van backtracking waarbij men over meerdere waarden tegelijk kan backtracken. Deze technieken versnellen het zoekproces in dat ze onnodige stappen detecteren en overslaan. Anderzijds bestaan er propagatie technieken die lokale consistentie garanderen met technieken zoals arc consistency, hyper-arc consistency en path-consistency die de domeinen van de variabelen proberen te verkleinen alvorens het zoekproces gestart wordt. 


\subsection{Interactive Configuration Problems}
In deze thesis ligt de focus op een specifieke groep van problemen binnen het domein van CSP's genaamd Interactieve Configuratie (IC) problemen. Constraint Satisfaction Problems zijn problemen waar een toekenning van domeinwaarden voor de variabelen dient gezocht te worden waarvoor geldt dat aan alle constraints voldaan is. IC problemen hebben hetzelfde doel, maar bij CSP's wordt er naar een toekenning gezocht d.m.v. recursieve zoektechnieken terwijl het in het geval can IC problemen een gebruiker is die handmatig stap per stap een geldige toekenning gaat proberen samenstellen. Om de gebruiker hierin bij te staan is er nood aan software, software die de gebruiker zo goed mogelijk bijstaat gedurende dit hele proces. Uit ervaring is gebleken dat het schrijven van software voor dit soort problemen geen gemakkelijke opgave is. Zo is aangetoond dat een imperatieve aanpak voor het beschrijven van de regels van een probleem vaak moeilijk is \citep{gelle1996interactive}. Reden hiervoor is dat de regels over de betreffende domeinkennis vaak verspreid zit in de software in codesnippets. Daarboven is onderhoud van zulke software een enorm moeilijke opgave waarbij de kleinste wijziging in de constraints kan leiden tot een volledige herwerking van de code. Een alternatief hiervoor is een daclaratieve aanpak. In declaratieve toepassingen kunnen de regels van een probleem beschreven worden d.m.v. logische constraints. Gelle en Weigel \cite{gelle1996interactive} laten niet alleen zien dat de declaratieve bescvhrijving van een probleem overzichtelijker en duidelijker is maar ook dat een wijziging in de constraints gemakkelijk aangepast kan worden. En hoewel er met een imperatieve aanpak optimalere algoritmen ontwikkeld kunnen worden, is aangetoond dat declaratieve methoden ook goede prestaties kunnen neerzetten \citep{vlaeminck2009logical}. 

\subsection{Knowledge Representation: Het IDP Systeem}
Er ondertussen vele declaratieve systemen verschenen (Prolog, Ant, Lisp, ...). Wat opvalt is dat elk systeem vaak samenhangt met \'{e}\'{e}n enkele specifieke vorm van inferentie. Dus afhankelijk van de gewenste inferentie zal een probleem opnieuw moeten beschreven worden in een ander systeem dat deze inferentie kan toepassen, ondanks dat de kennis telkens dezelfde is. Deze gedachtegang ligt aan de basis van concept van Knowledge Representation \citep{denecker2008building}. In het KR-paradigma wordt gesteld dat ongeacht de inferentie los staat van de kennis over een probleem. Deze kennis is niets meer dan een verzameling van informatie, maar met deze informatie kunnen meerdere vormen van inferentie toegepast worden. In dit onderzoek ligt de aandacht op IDP, een KR-systeem ontwikkeld aan de K.U. Leuven en voor het eerst voorgesteld in 2008. Het laat toe om de kennis over een probleem te beschrijven in FO(\textperiodcentered), dit is eerste orde logica maar uitgebreid met aggregaten, type definities en inductieve definities. 

\section{Probleemstelling}

\subsection{Individueel Studieprogramma}
Een ISP samentstellen valt onder deze categorie van Interactieve Configuratieproblemen. De gebruiker, in dit geval een toekomstige student wil een geldig ISP bekomen d.m.v. het selecteren van mogelijke opleidingsonderdelen. Maar de student kan niet zomaar elk opleidingsonderdeel selecteren, er zijn een heleboel regels waaraan de selectie moet voldoen. Deze regels zijn op zichzelf duidelijk en intu\"{i}tief, maar om een selectie vinden die aan alle regels voldoet kan mogelijk verwarrend zijn. Het huidige systeem voorziet weinig ondersteuning in het bijstaan van de gebruiker gedurende het selectieproces. Het is pas als je een selectie bevestigt dat het systeem controleert of deze ook effectief correct is volgens de regels. En hoewel het systeem wel weergeeft aan welke regels er (in geval van inconsistentie) niet voldaan is, moet de gebruiker zelf op zoek gaan naar de selectie die verantwoordelijk is verantwoordelijk is hiervoor. De regels van een geldig ISP nemen geen lessenrooster in acht, wat dus betekent dat lesmomenten voor verschillende opleidingsonderdelen kunnen overlappen. 

\section{Doel}
In deze thesis wil ik onderzoeken of de regels van het ISP effici\"{e}nt kunnen beschreven worden in FO(\textperiodcentered). De regels zijn voor elke opleiding anders. De vraag is of het mogelijk is om met \'{e}\'{e}n enkele theorie in IDP alle opleidingen correct te kunnen beschrijven. Belangrijk hierbij is dat de domeinen sterk kunnen verschillen tussen opleidingen, en dat de eventuele theorie hier mee om moet kunnen gaan. 

De K.U. Leuven heeft haar eigen systeem voor het samenstellen van het ISP. Een minpunt van dit systeem is de ondersteuning die het biedt tijdens het selectieproces. /*WAT ZIJN DE TEKORTKOMINGEN*/ Hier kan beter gedaan worden, daarom wil ik meer ondersteuning proberen bieden. In een zelf ontworpen Front-end met grafische user interface wil ik de volgende functionaliteiten integreren:
\begin{description}
\item[Automatisch invullen van gevolgen] Als de student een vak A kiest en hieruit volgt dat vak B ook gevolgt moet worden, dan is het de bedoeling dat het systeem dit automatisch invult zodat de student zich hier verder niets van hoeft aan te trekken. 
\item[Detectie van foutieve selectie] Deze functionaliteit is momenteel al aanwezig in het huidige systeem. Maar de detectie gebeurt pas bij de bevestiging van de selectie i.p.v. op het moment van de selectie zelf. En dat is wat ik zal proberen te integreren in de nieuwe front-end. En niet enkel dit, maar ook het effectief opsporen van de oorzaak zodat de gebruiker dit kan aanpassen.
\item[Geldig ISP laten genereren] Stel dat een student keuzes heeft gemaakt omtrend de vakken die hij/zij echt wil of niet wil volgen, maar de selectie is nog geen volledige oplossing. Dan kan de student vragen aan het systeem om de selectie verder in te vullen.
\item[Optimaal ISP laten genereren] Niet alleen moet het systeem een geldige oplossing kunnen genereren, maar ook de beste oplossing volgens een bepaalde criterium. Zo zou een student bijvoorbeeld graag een ISP willen waarbij de werklast zo goed mogelijk verdeeld is over beide semester. 
\item[Ongedaan maken van acties] Deze functionaliteit is terug te vinden in zowat de meeste moderne systemen. Als een student ontevreden is over zijn/haar recente keuzes, moet er de mogelijk zijn om deze ongedaan te kunnen maken. 
\item[Weergave van het lessenrooster] Voorheen had ik al vermeld dat studenten geen duidelijk overzicht hebben van het lessenrooster dat voortvloeit uit hun keuzes. Vakken kunnen lesmomenten hebben die mogelijk overlappen met die van andere vakken en dit kan voor een ongewenste verrassing zorgen eens het ISP bevestigd is. Het is mijn bedoeling om in de nieuwe front-end wel een eventueel lessenrooster weer te geven.
\end{description}
De meeste van deze functionaliteiten steunen op inferentie technieken die IDP aanbeidt zoals model expansie, minimizatie, propagatie etc. Dit wil zeggen dat om deze functionaliteiten te kunnen aanbieden, de onderliggende inferentie technieken effici\"{e}nt moeten werken. Herinner dat we te maken hebben met een IC probleem en dit vereist dat de reactietijd niet meer dan enkele seconden bedraagt. Ik zal dus moeten onderzoeken of inferentie goed verloopt binnen dit probleem.

Het opsporen van foutieve selecties is \'{e}\'{e}n ding, maar kunnen \emph{verklaren} wat er mis is met de selectie is een ander verhaal. 
Het domein van Conflict explanation bevat nog een heel aantal openstaande vragen. IDP zelf voorziet zelf de unsatcore, en methode die opzoek gaat naar de regels die (in geval van unsatisfiability) nooit waar gemaakt kunnen worden. Probleem is echter dat de output geformuleerd is in FO(\textperiodcentered), en een doorsnee gebruiker zal niet in staat zijn om hieruit iets te kunnen afleiden. Verscheidene andere technieken voor conflict explanation die zijn voorgelegd leggen de focus op het optimaliseren van de rekentijd \citep{amilhastre2002consistency} of het zoeken naar minimale correcties \citep{o2005generating}. Maar in verklaring geven in een formaat dat begrijpbaar is voor iedereen, daarvoor bestaat nog geen algemene techniek. En in deze thesis zet ik een stap in het ongewisse om op zoek te gaan naar oplossing.


\section{Dependencies}
\begin{description}
\item [FO(\textperiodcentered)] Het is de bedoeling om de mogelijkheden van IDP in actie te zien en met een proof of concept de prestaties van het systeem te testen. FO(\textperiodcentered) bezit een grote uitdrukkingskracht, bovenop eerste order logica bevat het ook taalelementen zoals aggregaten (sum, avg, min, max, ..) en inductieve definities. 
\item [Kivy] Bij de keuze van de programmeertaal en framework viel mijn oog op Kivy. Dit is een Python library met een hele waaier aan grafische elementen die elk uitgebreid geconfigureert kunnen worden. Hiervoor kan gebruik gemaakt worden van een door de uitgever ontwikkelde kv-language. Kivy applicaties werken cross-platform en de applicaties zijn event-driven gebruik makend van een centrale loop.
\item [JSON] Het gegevensformaat van JSON is ideaal om de domeinen van verschillende opleidingen in weer te geven. De domeinwaarden zitten gekoppeld aan een attribuut, dit maakt het gemakkelijk om structuur te parsen en zo het domein voor iederen opleiding in te lezen.
\end{description}
\chapter{Related Work}
\label{cha:relatedwork}

Het IDP systeem is reeds beschreven in verscheidene papers die de motivatie, opbouw en werking gedetailleerd uitleggen. Aan de basis liggen vooral de werken van \citep{de2014predicate} en \citep{de2014separating}. Beide papers vormen een basis en een introductie voor iedereen die zich wenst te verdiepen in IDP. Ze beschrijven de onderdelen van FO(\textperiodcentered) en hoe deze kunnen gebruikt worden om de regels van een probleem uit te drukken en hoe deze regels tenslotte gebruikt kunnen worden om meerdere vormen van inferentie te doen. 

\paragraph{Consistency Restoration and Explanations in Dynamic CSP's: Application to Configuration \cite{amilhastre2002consistency}}
In de klasse van interactieve configuratieproblemen is het belangrijk dat de reactietijd van de inferentie technieken laag ligt. Het probleem is echter dat IC problemen net zoals CSP's enorm complex zijn, wat maakt de sommige inferentie technieken intractable (onhandelbaar) zijn in het slechtste geval. In de paper lossen ze dit probleem op door een automaat te bouwen, deze is een compacte weergave van de oplossingsverzameling voor het probleem in kwestie. De voorwaarde is wel dat deze automaat vooraf in een \textit{off-line} compilatiefase wordt opgesteld, dit hoeft maar \'{e}\'{e}n maal te gebeuren. Gebruik makend van deze automaat toont de paper aan dat bepaalde technieken op een effici\"{e}nte en snelle manier kunnen berekend worden. Om hun woorden kracht bij te zetten hebben ze een configuratie probleem van Renault voor het configureren van auto's gebruikt om hun model te testen. De resultaten ervan zijn trouwens zeer indrukwekkend. 

IDP voorziet momenteel wel een techniek voor het opsporen van conflicten, maar kan geen minimale oplossingen berekenen om niet-satisfieerbaarheid op te lossen. De automaat samen met de technieken uit de paper kunnen dit wel, dit bovenop de voordelen van het gebruik van de automaat. Daarom heb ik gekozen voor hier verder op in te gaan en te onderzoeken of deze techniek in combinatie met IDP vruchten kan afwerpen.

\paragraph{Lowering the learning curve for declarative programming: a Python API for the IDP system \cite{vennekens2015lowering}}
Nog te vaak zijn programmeurs terughoudend als het aankomt op declaratief programmeren omwille van de leercurve die ermee gepaard gaat. Het doel van de auteur is om de moeilijkheidsgraad te verlagen in de hoop dat het de stap naar declaratieve systemen vergemakkelijkt. De auteur probeert dit te bereiken door een IDP KR-systeem te integreren in een Python API. Python is \'{e}\'{e}n van de meest gebruikte programmeertalen, en dankzij de API kan iedereen vertrouwd met Python gebruik maken van het het IDP KR-systeem zonder de syntax te moeten leren. Doch zijn declaratieve systemen inherent verschillend van imperatieve systemen, en iedereen die geen opleiding omtrent declaratieve systemen heeft genoten zal dit dan toch nog moeten inhalen. Bij studenten Computerwetenschappen is dit echter wel het geval aangezien het standaard onderdeel is van de opleiding. En de auteur hoopt met de API dat studenten sneller de stap zullen maken naar IDP, door ze de mogelijkheid te geven om dit te doen in de vaak vertrouwde Python omgeving.

\paragraph{Generating Corrective Explanations for Interactive Constraint Satisfaction \cite{o2005generating}}
In het veld van interactieve configuratieproblemen is het zo dat als de gebruiker een ongeldige combinatie van variabelen selecteert het systeem op zoek moet gaan naar de oorzaak hiervan. Anders gezegd gaat het een verklaring zoeken voor het probleem. Dit is natuurlijk handig voor de gebruiker om te achterhalen welke selectie hij moet aanpassen om het conflict op te lossen. Maar nog handiger zou zijn dat niet enkel de oorzaak wordt getoond, maar ook een (of meerdere) mogelijke oplossing(en). Oplossingen tot nu toe werden altijd gezien als de minimale set van variabelen waarvoor de geselecteerde waarde uit het domein ongedaan gemaakt dient te worden, zodanig dat de resterende selectie terug uitgebreid kan worden tot een geldig model. Hier heeft oplossing een geheel andere betekenis. Een oplossing is nog altijd een minimale set van variabelen, maar in plaats van enkel te zeggen dat de keuzes voor deze variabelen ongedaan moet worden gemaakt is er een waarde uit het domein gegeven van de variabelen waarvoor geldt dat ze deel uitmaakt van een geldige selectie. De paper stelt CORRECTIVEEXP voor, een systematisch algoritme voor het zoeken van \emph{minimal corrective explanations}. Een vergelijkende studie met reeds bestaande technieken op verscheidene grote problemen toont aan dat het algoritme zeer goede resultaten boekt. Dit is essentieel aangezien dat interactieve configuratie problemen een snelle reactietijd vereisen.

De inferentietaak die de auteur in zijn paper voorstelt maakt geen deel uit van IDP. Het is een heel interessant concept en zou in de toekomst mogelijk een handige toevoeging kunnen zijn aan het IDP systeem. In mijn zoektocht naar mogelijke technieken heb ik overwogen te kiezen voor deze techniek, maar heb uiteindelijk besloten om dit niet te doen en in plaats daarvan heb ik gekozen voor de implementatie volgens \citep{amilhastre2002consistency}.

\chapter{Concept}
\label{cha:concept}

\chapter{Implementation}
\label{cha:implementation}

\section{Knowledge base constructie}

\subsection{Theorie van het ISP}
Geen opleiding aan de KU Leuven is dezelfde, de opleidingsonderdelen verschillen net zoals de vakgroepen waar ze deel van uitmaken. Voor sommige vakgroepen ben je verplicht alle onderdelen op te nemen, terwijl je voor andere en minimaal (of zelfs maximaal) aantal studiepunten ervan moet opnemen. Kortom de regels binnen \'{e}\'{e}nzelfde opleiding zijn niet zo moeilijk om te beschrijven, maar om dit te veralgemenen en een theorie te cre\"{e}ren die voor alle opleidingen een geldig ISP kan beschrijven dat is een andere paar mouwen. Voor het gemak van de lezer noemen we opleidingsonderdelen vanaf nu vakken.

\subsubsection{Vakken}
Een vak is een simpel type dat altijd dezelfde eigenschappen heeft vertoond. Het heeft een unieke vakcode met daaraan een naam gekoppeld. Het telt een aantal studiepunten dat de werklast beschrijft.
Een opleiding kan kan bestaan uit \'{e}\'{e}n of meerdere fases, en voor elk vak is er bepaald in welke fase(s) van de opleiding je het kan volgen. let wel je kan een vak per opleiding maar \'{e}\'{e}n keer volgen. En tenslotte valt elk vak oftwel in het eerste, tweede of beide semesters (in dit laatste geval noemt men het een jaarvak).

\subsubsection{Vakgroeptypes}
Zo valt op dat we telkens dezelfde \textbf{types} van \emph{vakgroepen} tegenkomen. Deze types die steeds terugkomen in meerdere opleidingen hebben ongeacht de opleiding waar ze in voorkomen dezelfde regels die ermee gepaard gaan. Zo is elke opleiding op zich eeen vakgroep van het type \emph{opleiding}, het bevat vakken die je verplicht bent te volgen en keuzevakken. Het heeft een minimum (en maximum) aantal studiepunten dat de student verplicht moet opnemen. Een ander belangrijk type vakgroep is de (hoofd)specialisatie, verscheidene opleidingen geven de keuze tussen meerdere specialisaties. Verwacht wordt dat je van minstens 1 zo'n specialisatie alle verplichte vakken opneemt. Naast hoofdspecialisatie bestaat er ook het type verdere specialisatie, waarin de student verplicht wordt een bepaald aantal studiepunten op te nemen aan vakken uit deze of bepaalde andere vakgroepen. Vervolgens is er het type 'Algemeen vormende en onderzoeksondersteunende groep' vakgroep, dit is veruit het moeilijkste type groep om te beschrijven. De student moet opnieuw een bepaald aantal studiepunten aan vakken opnemen uit deze groep. Maar daarnaast gelden er ook specifieke regels voor verscheidene vakken die deel uitmaken ervan. En als laatse is er het type 'Bachelor verbredend pakket', waarin studenten die beginnen aan hun masteropleiding vakken moeten opnemen die ontbraken in hun bacheloropleiding.

Deze types zien we het vaakst voorkomen en kunnen ongeacht de opleiding met dezelfde set van regels beschreven worden. Het is mogelijk dat sommige types van vakgroepen niet telkens voorkomen, en dus de regels ook niet van kracht zijn. In een bacheloropleiding zal bijvoorbeeld nooit een vakgroep voorkomen van het type Bachelor verbrendend pakket. 

\subsection{Theorie van het lessenrooster}



\section{Front-End}

Om een zo goed mogelijke gebruikservaring aan te kunnen bieden hebben we een Front-end applicatie ontwikkeld voorzien van een Grafische Interface. Deze communiceert under the hood met de IDP knowledge base. Alle selecties gemaakt door de gebruiker worden vertaald naar FO(\textperiodcentered) en doorgespeeld aan IDP. Vervolgens voert IDP de gewenste vorm van inferntie uit en geeft het resultaat hiervan terug aan de Front-end die dan verwerkt en aan de gebruiker laat zien via de GUI.

\subsection{Grafische Gebruikersinterface}



\subsection{Communicatie tussen Front- en Back-end}
Niet alleen zijn is de syntax tussen IDP en Python compleet verschillend, het zijn ook twee verschillende klassen van programmeertalen respectievelijk declaratief en objectgeori\"{e}nteerd. Dus communicatie tussen de twee talen is niet vanzelfsprekend. Een Python API die toelaat IDP te gebruiken is reeds ontwikkeld \citep{vennekens2015lowering}. En hoewel deze API zeker de kloof tussen beide programmeertalen kleiner maakt, schiet ze op sommige vlakken toch te kort. Zo laat het bijvoorbeeld niet toe om aggregaties te te beschrijven. Daarom heb ik gekozen om ze niet te gebruiken en in plaats daarvan zelf de communicatie tussen beide te regelen. De theorie en vocabulair worden uitgelezen uit een tekstbestand en de structuur kan automatisch gegenereerd worden door de Vak en Vakgroep klassen in Python zelf. Een parser klasse steekt alles samen in een string en geeft dit door aan de IDP solver. Het resultaat wordt dan gefilterd (de nieuwe waarden worden eruit gehaald) en de veranderingen worden verwerkt. 

\section{Features}

\subsection{Propagatie}
Om het ISP samen te stellen moeten volgende, nog onbekende predicaten ingevuld worden door de gebruiker: Geselecteerd(Vak,Fase) en NietGeintereseerd(Vak). Bij elke stap (nieuwe selectie die de gebruiker maakt) wordt indien die selectie satisfieerbaar is, eventuele propagaties ook geselecteerd. 

\subsection{Model Expansie}
Als de gebruiker een aantal selecties heeft gedaan en voor de rest geen specifieke eisen meer heeft, kan hij vragen aan het systeem om het ISP verder te laten genereren. 

\subsection{Minimizatie}
Ook is het mogelijk om het systeem het beste ISP te laten genereren volgens bepaalde regels. Momenteel is het mogelijk om een ISP te zoeken dat zo weinig mogelijk werklast bevat (zo min mogelijk studiepunten). Of om de werklast zo goed mogelijk te verdelen over de beide semesters. 
Daarnaast is het ook mogelijk om de vakken zo te selecteren dat hun lesmomenten zo weinig mogelijk overlappen. 

\subsection{Selectieproces}
Elke keuze die de gebruiker maakt moet zorgvuldig afgehandeld worden willen we dat de selectie altijd satisfieerbaar is. De voorwaarde is dat als de gebruiker een keuze maakt die ervoor zorgt dat de theorie onsatisfieerbaar wordt, hij hier meteen op de hoogte van wordt gebracht en dit dient aanpassen. 

De werking is als volgt, bij elke nieuwe keuze wordt eerst gekeken of deze nieuwe selectie samen met alle voorgaande keuzes een satisfieerbare structuur vormen. Zo niet dan wordt de unsatstructure gezocht. Anders worden die eventueel nieuwe propagaties berekend. De nieuwe structuur wordt vergeleken met de voorgaande, om na te gaan of eventuele propagaties niet meer van kracht zijn. Als dit het geval is dan krijgt de gebruiker de keuze om ze te behouden. Dit hij dit dan wordt dit gezien als een selectie en dus begint het hele proces opnieuw. 

\begin{algorithm}
	\SetKwInOut{Input}{Input}
	\SetKwInOut{Output}{Output}
	\underline{function Selectieproces} $(\mathcal{U}\textsubscript{i})$ \;
	\Input{De nieuwe keuze van de gebruiker $\mathcal{U}\textsubscript{i}$}
	\Output{$\mathcal{U}\textsubscript{i}$, nieuwe propagaties $\mathcal{P}\textsubscript{i}$}
	$\mathcal{U} \leftarrow \mathcal{U}\textsubscript{1}  \cup$ ... $\cup  \mathcal{U}\textsubscript{i}$\;
	\eIf{Sat($\mathcal{U}$)}
		{
		$\mathcal{P}\textsubscript{i} \leftarrow$ Propagate($\mathcal{U}\textsubscript{1}$)\;
		
		}
		{
		$\mathcal{V} \leftarrow$ Unsat($\mathcal{U}$)\; 
		}
	\caption{Selectieproces}
\end{algorithm}

\subsection{Ongedaan Maken}
Het ongedaan kunnen maken van selecties is een van de aspecten waar meerdere strategi\"{e}en mogelijk zijn. Belangrijk op weten is dat er een duidelijk onderscheid gemaakt moet worden tussen de keuzes gemaakt door de gebruiker en de eventuele propagaties die hieruit volgen. Deze moeten dan ook voor elke tussenstap keurig bijgehouden worden in wat we vanaf nu een actie zullen noemen. Het is eveneens belangrijk om de originele waarden bij te houden, zodanig dat als men een actie ongedaan maakt de toestand terug kan gezet worden naar hoe ze voorheen was. Het voorgaande is allemaal redelijk vanzelfsprekend, maar de moeilijkheid zit hem in de verschillende mogelijke strategie\"{e}n. 

\paragraph{Strategie 1}
Een mogelijkheid is om simpelweg een actie te zien zoals een overgang van \'{e}\'{e}n consistente toestand naar een nieuwe consistente toestand. Dus elke actie weet hoe de selectie eruit zag voordat ze uitgevoerd was, en hoe de selectie eruit zag na afloop van de actie. Een actie ongedaan maken houdt dan simpelweg in dat je de nieuwe toestand terug vervangt door de oude, een beetje als een rollback operatie. De voorwaarde hier is dat enkel de laatste actie ongedaan gemaakt kan worden, alvorens mijn de voorlaatste ongedaan wil maken. 

\paragraph{Strategie 2}
Dat brengt ons bij het volgende, bij het ongedaan maken van een actie draaien we als het ware de tijd terug en wordt de huidige toestand vervangen door de voorgaande. Dit wil zeggen dat ook de propagaties uit de actie verdwijnen, en misschien wil de gebruiker dit niet. In plaats daarvan willen we bij het ongedaan maken van een operatie de gebruiker laten kiezen of de eventuele propagaties mogen geselecteerd blijven of dat ze ook ongedaan gemaakt mogen worden. Hier komen we het volgende probleem tegen. Stel dat het vak A geselecteerd is en hieruit volgt dat vak B ook gevolgd moet worden. In de volgende actie selecteert de gebruiker dat hij niet ge\"{i}ntereseerd is in A en hieruit volgt dat B niet gevolgd mag worden. De gebruiker komt terug op zijn beslissing en wil de laatste actie ongedaan maken. Dit wil zeggen dat de keuze van vak A van niet ge\"{i}ntereseerd terug gezet wordt naar wel geselecteerd. Hieruit volgt dat de propagatie over vak B niet meer van kracht is, en de gebruiker dus de keuze krijgt om deze ongedaan te maken of niet. Stel dat hij ervoor kiest de propagatie te behouden dan wordt het probleem onsatisfieerbaar. Dit is niet het enigste probleem dat zich voordoet met deze strategie. Iedere actie beschrijft hoe de toestand voor en na de actie. Als de gebruiker een actie ongedaan maakt maar ervoor kiest om de propagaties uit deze actie toch te behouden, dan wil dit dus zeggen dat het resultaat van de voorgaande actie niet meer klopt met de huidige situatie. In feite maken we dus niet heel de actie ongedaan wat het heel moeilijk maakt om een duidelijk overzicht te bieden en consistentie te garanderen. Dat brengt ons bij het volgende punt. We willen actie Ai ongedaan maken, de actie bevat Ui de keuze van de gebruiker en Pi de propagaties die eruit volgden. De gebruiker kiest ervoor om p1 $\in$ Pi te behouden. Ai wordt ongedaan gemaakt, maar nu wordt er een actie A'i toegevoegd waarin U'i ={p1} de nieuwe selectie van de gebruiker is en P'i de eventuele propagaties die hieruit volgen. 

\paragraph{Strategie 3}
Tenslotte is er nog een geheel andere optie, waarbij niet meer uit wordt gegaan van acties die men kan ongedaan maken. Een vak kan geselecteerd zijn in een fase, het kan zijn dat de gebruiker niet ge\"{i}nteresseerd is in een vak of het kan zijn dat de gebruiker nog geen keuze heeft gemaakt. Het idee van iets ongedaan maken bij deze strategie is het volgende: de gebruiker heeft geselecteerd dat hij/zij vak A wil volgen tijdens fase 1 van de opleiding. Later zegt de gebruiker dat hij zij dit niet meer wel, maar zegt verder specifiek niets over het vak. We hebben de volgende verzamelingen: Ci de vakken waar de gebruiker specifiek een keuze over heeft gemaakt, Pi de propagaties die volgen uit Ci en Ui de vakken waarover nog geen uitspraak is gedaan. Stel vak V $\in$ Cn-1 maar de gebruiker maakt de selectie ongedaan en V wordt toegevoegd aan Un. Dus Cn = Cn-1 - V. dan gaan we kijken wat Pn en Un zijn, en vervolgens gaan we zien wat de doorsnede is van Pn-1 en Un, is deze verzameling niet leeg, dan wil dit zeggen dat er propagaties zijn die niet langer bestaan, en de gebruiker moet kiezen of hij/zij ze wil behouden. Zo ja worden ze toegevoegd aan Cn+1 en begint het hele process opnieuw.

\section{Conflict Explanation}
Het is mogelijk dat de gebruiker een verkeerde keuze maakt waardoor de theorie nietmeer satisfieerbaar is. Simpel gezegd heeft de gebruiker een waarde gekozen voor een bepaalde variabele, zodanig dat dit ervoor zorgt dat \'{e}\'{e}n of meerdere regels uit de theorie niet meer waar kunnen worden gemaakt.

\begin{lstlisting}[mathescape]
A $\wedge$ B
$\neg$ A
\end{lstlisting}

In het voorbeeld, zegt de theorie ons dat A en B beide waar moeten zijn. We zeggen dat A niet waar is. Door deze invulling kan de regel nooit waar worden ongeachte de waarde van B. De theorie is dus onsatisfieerbaar. 

Conflict explanation wil zeggen een verklaring zoeken waarom een bepaalde selectie door de gebruiker dit veroorzaakt. 

IDP gebruikt momenteel twee technieken om oorzaken van onsatisfieerbaarheid op te sporen. De unsatstructure spoort de set van variabelen op die het probleem veroorzaken. Vervolgens is er de unsatcore, die zoekt naar de kleinste set van regels uit de theorie die niet waar gemaakt kunnen worden.

\subsubsection{Unsatstructure}
De unsatstructure is een effici\"{e}nte tool om aan de gebruiker te kunnen meedelen, welke van de voormalige selecties problemen veroorzaken. In ons voorbeeld van het ISP kan de gebruiker enkel de keuze maken om een dag te volgen in een bepaalde fase (te kiezen uit de lijst van mogelijke waarden) of kan hij/zij aangeven om een vak niet te volgen. 
Bij een slechte selectie zal de unsatstructure de vakken oplijsten samen met de selectie die de gebruiker ervoor gemaakt heeft, en dit enkel voor de vakken waarvoor de selectie bijdragen tot de onsatisfieerbaarheid. De gebruiker krijgt hier dan de mogelijkheid om voor deze vakken de voorheen gemaakte keuze ongedaan te maken of te veranderen. 

\subsubsection{Reified Constraints}
Wat opvalt is dat de voorgaande techniek enkel een verzameling selecties teruggeeft die bijdragen tot het probleem. Maar verdere uitleg over welke regel(s) uit de theorie niet meer waar gemaakt kunnen worden en waarom wordt niet gegeven. Dus de gebruiker kan niet weten wat er moet veranderen. 
Herinner de unsatcore zoekt achter de kleinste set van regels uit de theorie die niet meer waar gemaakt kunnen worden. En hoewel dit hetgene is dat we willen weten, is er toch nog een probleem. 
De IDP syntax is voor de doorsnee programmeur goed te lezen, maar voor een doorsnee gebruiker die niets van programmeren afweet zal dit ongetwijfeld als chinees overkomen. Neem als voorbeeld onderstaande regel, elke student computerwetenschappen zal vrij snel kunnen achterhalen wat de regel inhoud. Maar personen zonder deze achtergrond zullen dit niet kunnen verstaan.
\begin{lstlisting}[mathescape, caption=IDP Rule Example, frame=topline/bottomline]
$\forall$vg[VakGroep] : IsType(vg,AVO) $\Rightarrow$ GesAantalStupunVakGr(vg) 
= sum{v[Vak], sp[Studiepunten],f[Fase] : InVakGroep(v,vg) 
$\wedge$ Geselecteerd(v,f) $\wedge$ AantalStudiepunten(v)=sp : sp }.
\end{lstlisting}
We proberen dit probleem deels te overbruggen d.m.v. reified constraints. Hierbij koppelen we de waarheidswaarde van een regel (constraint) uit de theorie aan een booleaanse variabele. Nemen we terug onze regel uit voorbeeld 1, daarvoor plaatsen we de Booleaanse variabele C toe gevolgd door een equivalentie. Wat dit wil zeggen is dat de waarheidswaarde voor en na de equivalentie dezelfde moet zijn. Dus als de regel uit voorbeeld 1 niet voldaan is, en dus 'false' is dan moet C ook 'false' zijn. 
\begin{lstlisting}[mathescape]
C $\Leftrightarrow$ A $\wedge$ B
$\neg$ A
\end{lstlisting}
Op deze manier hoeven we enkel na te gaan of C true of false is om te weten te komen of de regel waar is of niet. Hiermee is de gebruiker natuurlijk nog niets wijzer geworden, dus wat is nu net het nut van dit soort constraints? Voor elke regel uit de theorie hebben we een uitleg in natuurlijke taal geschreven die beschrijft wat er mis is als de regel niet consistent is. Daarna moeten we simpelweg voor elke regel controleren wat de waarheidswaarde is van de booleaanse variabele voor de equivalentie. Is deze false, dan is de regel dus inconsistent en krijgt de gebruiker de uitleg te zien van deze regel in natuurlijke taal. 

Hoewel dit een zeer gemakkelijke en generische manier is om aan conflict explanation te doen, zijn de mogelijkheden van deze techniek toch gelimiteerd en niet altijd voldoende toereikend. Ten eerste laat deze methode enkel toe om inconsistente regels op te sporen. Een regel is inconsistent als ze niet voldaan is gegeven de huidige selectie. Terwijl een regel die nooit waar gemaakt kan worden gegeven de huidige selectie onsatisfieerbaar is. En ten tweede is de uitleg vaak oppervlakkig en niet specifiek genoeg om duidelijk te kunnen aanwijzen welke selectie juist verantwoordelijk is voor de inconsistentie. 
Desalniettemin geeft deze techniek de gebruiker redelijk goed zicht op het probleem en dragen ze toch bij tot het maken van juiste beslissingen. Een voordeel is dat zodra er geen inconsistente regels meer weergegeven zijn de gebruiker weet dat de gemaakte selectie consistent is met de theorie. 


\chapter{Evaluation}
\label{cha:evaluation}
De bedoeling van dit onderzoek is om de regels van het ISP te kunnen beschrijven in IDP, zodanig dat ze voor elke opleiding aan de KU Leuven een correct model kunnen genereren. Daarnaast is het de bedoeling om het lessenrooster mee in rekening te brengen. Aangezien het mogelijk is dat vakken overlappen, is het toch belangrijk dat de student op de hoogte is van deze informatie. Met deze regels willen we natuurlijk inferentie kunnen doen, en een belangrijke vereiste van een Interactief configuratieprobleem zoals een ISP samenstellen is een snelle respons. De inferentietaken moeten dusdanig snel afgehandeld kunnen worden dat de reactietijd hooguit enkel seconden bedraagt. En tenslotte is er conflict explanation, waar bij een foute selectie de gebruiker een specifieke uitleg hoort te krijgen wat er mis is, waarom en hoe het opgelost kan worden. Dit alles in een zo natuurlijk mogelijke taal die de gebruiker gamakkelijk kan verstaan. 

\begin{description}
\item [Theorie van het ISP] Geen twee opleidingen zijn dezelfde, en toch is het belangrijk dat ongeacht deze verschillen we telkens dezelfde set van regels kunnen gebruiken zonder daarin te moeten gaan aanpassen. Om de omvang van het project haalbaar te houden voor \'{e}\'{e}n persoon, is het domein van opleidingen beperkt gebleven tot die van de computerwetenschappen en informatica. Hiervoor ben ik erin geslaagd en theorie op te stellen die zonder probleem eender welk van deze opleidingen kan beschrijven. Hoewel de naamgeving vaak verschilt per opleiding, komen toch vaak structeren voor die dezelfde eigenschappen vertonen. Dus de echte uitdaging is deze structuren vinden, niet de regels ervoor schrijven. Voor een klein sub-domein binnen de opleidingen van de KU Leuven is het dus mogelijk een theorie te vinden. 

\item [Het lessenrooster] 

\item [Inferentie] Met behulp van de interface kan de gebruiker een ISP samenstellen. Het is de bedoeling om de gebruiker bij te staan in dit proces, dit door gevolgen van bepaalde keuzes door te voeren, een onvolledige selectie te vervolledigen, een optimaal ISP samen te stellen volgens een bepaald criterium, foute keuzes te detecteren en de gebruiker hiervan op de hoogte te brengen etc. En niet te vergeten, al deze processen moeten in real-time gebeuren en dus zeer snel afgehandeld kunnen worden. Binnen het domein van de opleidingen, ben ik tot de conclusie gekomen dat zowat al deze inferentie taken snel zeer snel en effici\"{e}nt uitgevoerd worden. De reactietijd, meegerekend het genereren van de IDP text file en het resultaat terug ontcijferen bedraagt in bijna alle gevallen minder dan 1 seconde. Zelfs de resultaten van minimizatie opdrachten geven goede resultaten terug met een maximale rekentijd van ongeveer 10 seconden. 

\item [Conflict Explanation] Als de gebruiker een keuze maakt waardoor de regels onsatisfieerbaar worden, dan is het de bedoeling dat de oorzaak hiervan opgespoord wordt en de gebruiker hierover wordt ingelicht. Via de unsatstructure van IDP krijgt de gebruiker een lijst te zien met vakken waarvoor hij/zij een keuze heeft gemaakt die de onsatisfieerbaarheid veroorzaken. Het is in deze lijst dat de gebruiker keuzes kan veranderen of ongedaan maken om det probleem op te lossen. Enkel een lijst tonen is natuurlijk niet genoeg, als er geen uitleg gegeven wordt waarom de huidige keuze fout is kan je ook niet weten wat je moet veranderen om het terug op te lossen. 

\end{description}
\chapter{Conclusion}
\label{cha:conclusion}



% If you have appendices:
\appendixpage*          % if wanted
\appendix
\chapter{The First Appendix}
\label{app:A}

% ... and so on until
\chapter{The Last Appendix}
\label{app:n}
Appendices are numbered with letters, but the sections and subsections use
arabic numerals, as can be seen below.

\section{Lorem 20-24}
\lipsum[20-24]

\section{Lorem 25-27}
\lipsum[25-27]

%%% Local Variables: 
%%% mode: latex
%%% TeX-master: "thesis"
%%% End: 


\backmatter
% The bibliography comes after the appendices.
% You can replace the standard "abbrv" bibliography style by another one.
\bibliography{references}

\end{document}

%%% Local Variables: 
%%% mode: latex
%%% TeX-master: t
%%% End: 

\chapter{Evaluation}
\label{cha:evaluation}
De bedoeling van dit onderzoek is om de regels van het ISP te kunnen beschrijven in IDP, zodanig dat ze voor elke opleiding aan de KU Leuven een correct model kunnen genereren. Daarnaast is het de bedoeling om het lessenrooster mee in rekening te brengen. Aangezien het mogelijk is dat vakken overlappen, is het toch belangrijk dat de student op de hoogte is van deze informatie. Met deze regels willen we natuurlijk inferentie kunnen doen, en een belangrijke vereiste van een Interactief configuratieprobleem zoals een ISP samenstellen is een snelle respons. De inferentietaken moeten dusdanig snel afgehandeld kunnen worden dat de reactietijd hooguit enkel seconden bedraagt. En tenslotte is er conflict explanation, waar bij een foute selectie de gebruiker een specifieke uitleg hoort te krijgen wat er mis is, waarom en hoe het opgelost kan worden. Dit alles in een zo natuurlijk mogelijke taal die de gebruiker gamakkelijk kan verstaan. 

\begin{description}
\item [Theorie van het ISP] Geen twee opleidingen zijn dezelfde, en toch is het belangrijk dat ongeacht deze verschillen we telkens dezelfde set van regels kunnen gebruiken zonder daarin te moeten gaan aanpassen. Om de omvang van het project haalbaar te houden voor \'{e}\'{e}n persoon, is het domein van opleidingen beperkt gebleven tot die van de computerwetenschappen en informatica. Hiervoor ben ik erin geslaagd en theorie op te stellen die zonder probleem eender welk van deze opleidingen kan beschrijven. Hoewel de naamgeving vaak verschilt per opleiding, komen toch vaak structeren voor die dezelfde eigenschappen vertonen. Dus de echte uitdaging is deze structuren vinden, niet de regels ervoor schrijven. Voor een klein sub-domein binnen de opleidingen van de KU Leuven is het dus mogelijk een theorie te vinden. 

\item [Het lessenrooster] 

\item [Inferentie] Met behulp van de interface kan de gebruiker een ISP samenstellen. Het is de bedoeling om de gebruiker bij te staan in dit proces, dit door gevolgen van bepaalde keuzes door te voeren, een onvolledige selectie te vervolledigen, een optimaal ISP samen te stellen volgens een bepaald criterium, foute keuzes te detecteren en de gebruiker hiervan op de hoogte te brengen etc. En niet te vergeten, al deze processen moeten in real-time gebeuren en dus zeer snel afgehandeld kunnen worden. Binnen het domein van de opleidingen, ben ik tot de conclusie gekomen dat zowat al deze inferentie taken snel zeer snel en effici\"{e}nt uitgevoerd worden. De reactietijd, meegerekend het genereren van de IDP text file en het resultaat terug ontcijferen bedraagt in bijna alle gevallen minder dan 1 seconde. Zelfs de resultaten van minimizatie opdrachten geven goede resultaten terug met een maximale rekentijd van ongeveer 10 seconden. 

\item [Conflict Explanation] Als de gebruiker een keuze maakt waardoor de regels onsatisfieerbaar worden, dan is het de bedoeling dat de oorzaak hiervan opgespoord wordt en de gebruiker hierover wordt ingelicht. Via de unsatstructure van IDP krijgt de gebruiker een lijst te zien met vakken waarvoor hij/zij een keuze heeft gemaakt die de onsatisfieerbaarheid veroorzaken. Het is in deze lijst dat de gebruiker keuzes kan veranderen of ongedaan maken om det probleem op te lossen. Enkel een lijst tonen is natuurlijk niet genoeg, als er geen uitleg gegeven wordt waarom de huidige keuze fout is kan je ook niet weten wat je moet veranderen om het terug op te lossen. 

\end{description}
\chapter{Evaluation}
\label{cha:evaluation}
De bedoeling van dit onderzoek is om de regels van het ISP te kunnen beschrijven in IDP, zodanig dat ze voor elke opleiding aan de KU Leuven een correct model kunnen genereren. Daarnaast is het de bedoeling om het lessenrooster mee in rekening te brengen. Aangezien het mogelijk is dat vakken overlappen, is het toch belangrijk dat de student op de hoogte is van deze informatie. Met deze regels willen we natuurlijk inferentie kunnen doen, en een belangrijke vereiste van een Interactief configuratieprobleem zoals een ISP samenstellen is een snelle respons. De inferentietaken moeten dusdanig snel afgehandeld kunnen worden dat de reactietijd hooguit enkel seconden bedraagt. En tenslotte is er conflict explanation, waar bij een foute selectie de gebruiker een specifieke uitleg hoort te krijgen wat er mis is, waarom en hoe het opgelost kan worden. Dit alles in een zo natuurlijk mogelijke taal die de gebruiker gamakkelijk kan verstaan. 

\begin{description}
\item [Theorie van het ISP] Geen twee opleidingen zijn dezelfde, en toch is het belangrijk dat ongeacht deze verschillen we telkens dezelfde set van regels kunnen gebruiken zonder daarin te moeten gaan aanpassen. Om de omvang van het project haalbaar te houden voor \'{e}\'{e}n persoon, is het domein van opleidingen beperkt gebleven tot die van de computerwetenschappen en informatica. Hiervoor ben ik erin geslaagd en theorie op te stellen die zonder probleem eender welk van deze opleidingen kan beschrijven. Hoewel de naamgeving vaak verschilt per opleiding, komen toch vaak structeren voor die dezelfde eigenschappen vertonen. Dus de echte uitdaging is deze structuren vinden, niet de regels ervoor schrijven. Voor een klein sub-domein binnen de opleidingen van de KU Leuven is het dus mogelijk een theorie te vinden. 

\item [Het lessenrooster] 

\item [Grafische Interface] De GUI dient als prototype voor een mogelijke versie die de student kan gebruiken in de toekomst. Daarin moet het minstens de functionaliteit bieden van het huidig systeem, maar met een aantal verbeteringen. Eerst en vooral is het de bedoeling dat de student de verschillende inferentietechnieken van de IDP back-end kan aanroepen om hulp te bieden bij het selectieproces. Het huidige systeem geeft enkel bij bevestiging terug of de selectie al dan niet correct is  volgens de regels. De nieuwe GUI doet dit in real-time op het moment van de foutieve selectie en weergeeft de oorzaak van het probleem. Naast dit alles is het ook de bedoeling om in het nieuwe systeem het lessenrooster te betrekken. De student krijgt een weekoverzicht te zien van de lessen voor de vakken die  hij/zij geselecteerd heeft. Zo kan er worden gecontroleerd of de lessen mogelijk overlappen. Daarbij is het ook mogelijk om IDP een ISP te laten samenstellen waarbij de lessen zo min mogelijk overlappen. De ontwikkeling van de GUI was zeer kostelijk, en heeft langer geduurd dan eerst gehoopt. De hoge configureerbaarheid van de elementen heeft tot gevolg dat het meer tijd kost om ze correct te implementeren. Hoewel de kv-language dit process vergemakkelijkt ligt de tijdsduur nog altijd hoger dan verwacht. Een van de belangrijkste criteria waar ik niet naar heb gekeken bij het kiezen van het framework is KIS (Keep It Simple). En dit zal ik zeker niet vergeten in de toekomst. 

\item [Inferentie] Met behulp van de interface kan de gebruiker een ISP samenstellen. Het is de bedoeling om de gebruiker bij te staan in dit proces, dit door gevolgen van bepaalde keuzes door te voeren, een onvolledige selectie te vervolledigen, een optimaal ISP samen te stellen volgens een bepaald criterium, foute keuzes te detecteren en de gebruiker hiervan op de hoogte te brengen etc. En niet te vergeten, al deze processen moeten in real-time gebeuren en dus zeer snel afgehandeld kunnen worden. Binnen het domein van de opleidingen, ben ik tot de conclusie gekomen dat zowat al deze inferentie taken snel zeer snel en effici\"{e}nt uitgevoerd worden. De reactietijd, meegerekend het genereren van de IDP text file en het resultaat terug ontcijferen bedraagt in bijna alle gevallen minder dan 1 seconde. Zelfs de resultaten van minimizatie opdrachten geven goede resultaten terug met een maximale rekentijd van ongeveer 10 seconden. 

\item [Conflict Explanation] Als de gebruiker een keuze maakt waardoor de regels onsatisfieerbaar worden, dan is het de bedoeling dat de oorzaak hiervan opgespoord wordt en de gebruiker hierover wordt ingelicht. Via de unsatstructure van IDP krijgt de gebruiker een lijst te zien met vakken waarvoor hij/zij een keuze heeft gemaakt die de onsatisfieerbaarheid veroorzaken. Het is in deze lijst dat de gebruiker keuzes kan veranderen of ongedaan maken om det probleem op te lossen. Enkel een lijst tonen is natuurlijk niet genoeg, als er geen uitleg gegeven wordt waarom de huidige keuze fout is kan je ook niet weten wat je moet veranderen om het terug op te lossen. 

Reified constraint bieden extra informatie en dienen als een helpende hand om de gebruiker door het selectieprocess te loodsen. Deze simpele generische manier om inconsistente regels op te sporen en te verklaren in natuurlijke taal heeft zeker zijn voordelen. De hoeveelheid werk vereist om deze functionaliteit te implementeren is zeer weinig. Maar de functionaliteit van deze techniek is echter beperkt. Het laat toe om inconsistente regels op te sporen in tegenstelling tot wat we zoeken namelijk de oorzaken van onsatisfieerbaarheid. Daarbij is de uitleg die voorzien wordt bij een inconsistente regel mogelijk niet toereikend voor de gebruiker, zo kan er niet aangewezen worden voor welke instantie van de variabelen een bepaalde regel inconsistent is, en het concept dat de regel beschrijft is vaak heel algemeen waardoor de gebruiker er moeilijk iets uit kan afleiden. Ondanks deze minpunten is deze techniek zeker nuttig in de zin dat het de gebruiker in de juiste richting kan leiden tijdens het selectieproces, en dit voor een lage implementatiekost.

Amilhastre paper
\end{description}
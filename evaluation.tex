\chapter{Evaluation}
\label{cha:evaluation}
De bedoeling van dit onderzoek is om na te gaan of de regels van het ISP in IDP beschreven kunnen worden. En hoe algemeen deze regels gemaakt kunnen worden zodanig dat dezelfde theorie meerdere opleidingen kan beschrijven. Daarnaast is het de bedoeling om het lessenrooster mee in rekening te brengen. Aangezien het mogelijk is dat vakken overlappen, is het toch belangrijk dat de student op de hoogte is van deze informatie. Met deze regels willen we natuurlijk inferentie kunnen doen, en een belangrijke vereiste van een Interactief configuratieprobleem zoals een ISP samenstellen is een snelle respons. De inferentietaken moeten dusdanig snel afgehandeld kunnen worden dat de reactietijd hooguit enkel seconden bedraagt. En tenslotte is er nog conflict explanation, waarbij ik wil nagaan wat de mogelijkheden zijn binnen ISP en of andere externe technieken \cite{amilhastre2002consistency} een alternatief kunnen bieden.

\subsection{Theorie van het ISP}
Geen twee opleidingen zijn dezelfde, en toch zou het gemakkelijk zijn als ongeacht deze verschillen we telkens dezelfde set van regels kunnen gebruiken zonder daarin te moeten gaan aanpassen. Om de omvang van het project haalbaar te houden voor \'{e}\'{e}n persoon, is het domein van opleidingen beperkt gebleven tot die van de computerwetenschappen en informatica. Hiervoor ben ik erin geslaagd en theorie op te stellen die zonder probleem eender welk van deze opleidingen kan beschrijven. Hoewel de naamgeving vaak verschilt per opleiding, komen toch vaak structeren voor die dezelfde eigenschappen vertonen. Dus de echte uitdaging is deze structuren vinden, niet de regels ervoor schrijven. Voor een klein sub-domein binnen de opleidingen van de KU Leuven is het dus mogelijk een theorie te vinden. 

\subsection{Front-End Applicatie} 
Het hoofddoel van de Front-End (incl. Grafische User-Interface) is het simuleren van een mogelijke toekomstige applicatie die effectief gebruikt kan worden door studenten en om gemakkelijk de inferentie technieken van het IDP systeem te kunnen testen in een real-time setting. Momenteel bestaat er al een web applicatie waarmee een student een ISP kan samenstellen. Deze kan als maatstaf dienen in de beoordeling van het nieuwe systeem. Er kan namelijk pas gesproken worden van vooruitgang als de nieuwe toepassing betere prestaties levert en uitgebreidere functionaliteit biedt ten opzichte van de reeds bestaande standaard. Hierbij ligt de focus op (i) het verschil in functionaliteiten (welke functionaliteit is aanwezig in welke applicatie?) en (ii) het niveau van ondersteuning dat ze aanbieden. Criteria zoals design en platform zullen hier niet aan bod komen aangezien deze niet tot de essentie van dit onderzoek behoren. Prestaties van de verschillende inferentie technieken staan besproken in een verder hoofdstuk. 

De huidige web applicatie van de K.U. Leuven biedt een basispakket van ondersteuning aan, essentieel voor de student om zijn/haar ISP te kunnen vervolledigen. Het laat de student toe om te kiezen welke opleidingsonderdelen van de gekozen opleiding hij/zij van plan is te volgen. De selectie kan vervolgens gecontroleerd worden op correctie, m.a.w. het systeem zal nagaan of ze voldoet aan alle regels van het ISP.  En als de selectie niet voldoet aan deze regels zal het systeem dit melden en tevens ook uitleg voorzien waarom de selectie niet voldoet aan de regels. Deze functionaliteiten zitten ook allemaal inbegrepen in de nieuwe applicatie, samen met nog een aantal nieuwe functies. Hiermee kan ik veilig stellen dat het aanbod van functionaliteit in mijn nieuwe applicatie zeker uitgebreider is dan dat van het huidige systeem. 
Naast het basispakket is het ook mogelijk voor de gebruiker om een parti\"{e}le selectie verder te laten invullen door het systeem tot het een volledig geldige selectie is volgens de regels van het ISP, het is zelfs mogelijk om een optimaal ISP te laten opstellen volgens een bepaald criterium. Gedurende het selectieproces is het mogelijk dat de keuzes van gebruiker ervoor zorgen dat bepaalde opleidingsonderdelen geselecteerd moeten worden als een gevolg hiervan, het systeem zal deze keuzes automatisch invullen (propageren) zodat de gebruiker zich hier geen zorgen over moet maken. Elke nieuwe stap in het selectieproces wordt bijgehouden in een geschiedenis van acties, als de gebruiker niet tevreden is met de selectie en deze wil wissen dan kan dit door de acties ongedaan te maken. Wanneer de student een vak selecteert dan zal het systeem een partieel lessenrooster genereren, dit is handig om te controleren of mogelijk geen lesmomenten overlappen met elkaar. 

\begin{table}[]
\centering
\caption{Functionaliteiten}
\label{my-label}
\begin{tabular}{|l|l|l|}
\hline
 & Web applicatie & Nieuwe applicatie \\ \hline
Handmatige selectie & \checkmark & \checkmark \\ \hline
Controle correctheid & \checkmark & \checkmark \\ \hline
Verklaring v. incorrectheid & \checkmark & \checkmark \\ \hline
ISP genereren &  & \checkmark \\ \hline
Optimaal ISP genereren &  & \checkmark \\ \hline
Propagatie &  & \checkmark \\ \hline
Ongedaan maken & \checkmark & \checkmark \\ \hline
Geschiedenis v. Acties &  & \checkmark \\ \hline
Weergave lessenrooster &  & \checkmark \\ \hline
\end{tabular}
\end{table}

\subsection{Inferentie} 
Met behulp van de interface kan de gebruiker een ISP samenstellen. Het is de bedoeling om de gebruiker bij te staan in dit proces, dit door gevolgen van bepaalde keuzes door te voeren, een onvolledige selectie te vervolledigen, een optimaal ISP samen te stellen volgens een bepaald criterium, foute keuzes te detecteren en de gebruiker hiervan op de hoogte te brengen etc. En niet te vergeten, al deze processen moeten in real-time gebeuren en dus zeer snel afgehandeld kunnen worden. Binnen het domein van de opleidingen, ben ik tot de conclusie gekomen dat zowat al deze inferentie taken snel zeer snel en effici\"{e}nt uitgevoerd worden. De reactietijd, meegerekend het genereren van de IDP text file en het resultaat terug ontcijferen bedraagt in bijna alle gevallen minder dan 1 seconde. Zelfs de resultaten van minimizatie opdrachten geven goede resultaten terug met een maximale rekentijd van ongeveer 10 seconden. 

\subsection{Conflict Explanation} 
Als de gebruiker een keuze maakt waardoor de regels niet-satisfieerbaar worden, dan is het de bedoeling dat de oorzaak hiervan opgespoord wordt en de gebruiker hierover wordt ingelicht. 

\subsubsection{Unsatstructure}
Via de unsatstructure van IDP krijgt de gebruiker een lijst te zien met vakken waarvoor hij/zij een keuze heeft gemaakt die de onsatisfieerbaarheid veroorzaken. Het is in deze lijst dat de gebruiker keuzes kan veranderen of ongedaan maken om det probleem op te lossen. Enkel een lijst tonen is natuurlijk niet genoeg, als er geen uitleg gegeven wordt waarom de huidige keuze fout is kan je ook niet weten wat je moet veranderen om het terug op te lossen. 

\subsubsection{Reified Constraints}
De techniek van reified constraints biedt extra informatie en kan dienen als een helpende hand om de gebruiker door het selectieproces te loodsen. Deze simpele generische manier om niet-satisfieerbare regels op te sporen en te verklaren in natuurlijke taal heeft zeker zijn voordelen. De implementatiekost ligt zeer laag, maar de functionaliteit van deze techniek is helaas ook beperkt. Door de techniek te combineren met de unsatstructure is het mogelijk om niet-satisfieerbare regels effici\"{e}nt op te sporen. En door aan elke regel een uitleg in natuurlijk taal te koppelen (deze hoeft slechts \'{e}\'{e}nmalig per regel opgesteld te worden) kan het systeem de gebruiker een beter begrijpbare ondersteuning bieden door precies te zeggen welke regel(s) er verbroken is (zijn) en te specifi\"{e}ren hoe dit zou kunnen gebeurd zijn. Maar deze uitleg is zeer statisch, dus als dezelfde regel door verschillende keuzes verbroken wordt zal de uitleg in beide gevallen dezelfde zijn. In vele gevallen is deze informatie duidelijk en specifiek genoeg, maar in de implementatie staat ook een voorbeeld vermeld waarbij de uitleg niet veel extra hulp biedt omdat de regel die ze beschrijft te algemeen is. Sinds het de bedoeling is van dit onderzoek om een zo algemeen mogelijke theorie te ontwikkelen kan de vraag gesteld worden hoe nuttig het is om deze techniek te gebruiken in combinatie met dit concept. Momenteel is de scope van opleidingen nog z\'{e}\'{e}r beperkt en zelfs dan bestaan er al regels die te algemeen zijn een duidelijk beeld te kunnen scheppen van wat er precies mis zou kunnen zijn. In een poging om de theorie nog algemener te maken zodat nog meer opleidingen beschreven kunnen worden, bleken de regels zo algemeen te worden dat een beschrijving in natuurlijke taal niet langer nuttig was. Dus waarschijnlijk zal er een afweging moeten gemaakt worden tussen een algemene theorie die zoveel mogelijk opleidingen kan beschrijven of duidelijke uitleg kunnen bieden ten koste van de uitdrukkingskracht van de theorie.

\subsubsection{Amilhastre}
Sinds de automaat een voorstelling is van alle mogelijke oplossingsverzamelingen, moeten deze eerst allemaal berekend worden. IDP kan dit doen door model expansion uit te voeren en alle oplossingen weer te geven. Dit is een kostelijke operatie, maar kan als onderdeel gezien worden van het opstellen van de automaat. En deze bewerking wordt verwacht dat ze enige tijd in beslag kan nemen. 

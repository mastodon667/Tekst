\chapter{Conclusion}
\label{cha:conclusion}
\paragraph{Proof of concept}
De eerste onderzoeksvraag die gesteld werd in dit onderzoek was de vraag of het mogelijk was om met behulp van IDP een theorie te ontwikkelen voor het ISP probleem en belangrijker nog of we effici\"{e}nt inferentie kunnen doen met deze theorie. De voordelen van de uitdrukkingskracht van FO(\textperiodcentered) zijn al vaker aangehaald \cite{van2016kb} \cite{de2014predicate} in het verleden. En voor het huidige probleem houden deze beweringen opnieuw stand, sinds ik met behulp van FO(\textperiodcentered) erin geslaagd ben om een theorie te ontwikkelen waarin de regels van het ISP vervat zitten. Deze theorie is zelfs in staat om meerdere opleidingen binnen het domein van informatica en computerwetenschappen te beschrijven. 
In IDP en het concept van kennisbank-systemen is er een strikte scheiding tussen kennis en inferentie. De theorie wordt slechts \'{e}\'{e}nmaal beschreven en deze laten toe om meerdere vragen te beantwoorden d.m.v. veschillende vormen van inferentie toe te passen op de theorie. Dit is enkel nuttig als deze inferentie technieken effici\"{e}nt en snel werken, en in dit geval snel genoeg om te kunnen voldoen aan de eisen van een interactief configuratie probleem. Het testen van deze technieken werd gedaan met behulp van de zelf ontwikkelde front-end applicatie. Voor alle gebruikte technieken bleken de reactiesnelheden te voldoen aan de vereisten van een interactief configuratieprobleem. Zelfs voor de meest complexe bewerkingen (bv. minimizatie) boekt het IDP systeem goede resultaten. Dus met dit proof of concept wordt opnieuw de kracht van FO(\textperiodcentered) getoond samen met de prestaties van de inferentiemethoden binnen het paradigma van kennis representatie.

\paragraph{Conflict explanation}
Meerdere onderzoeken hebben de voordelen van declaratieve methoden \cite{gelle1996interactive} en specifiek die van KR-systemen \cite{de2014predicate} \cite{denecker2008building} \cite{van2016kb} \cite{vlaeminck2009logical} ten opzichte van imperatieve strategie\"{e}n al aangehaald. Dit betekent niet dat een declaratieve aanpak op alle vlakken beter is, of dat alle vragen rond declaratieve systemen beantwoord zijn. Binnen het domein van conflict explanation zijn er reeds verschillende technieken ontworpen voor het zoeken naar, verklaren en oplossen van conlict situaties binnen IC problemen. Elke techniek hangt samen met zijn eigen concept van wat een oplossing juist is en hoe deze geformuleerd hoort te worden. In deze paper heb ik een aantal interessante technieken ge\"{i}mplementeerd en de resultaten ervan beschreven. De eerste, reified constraints maken enkel gebruik van de tools beschikbaar in IDP terwijl de tweede methode technieken combineert binnen IDP met die uit het werk van \citep{amilhastre2002consistency}. Reified constraints, is een techniek met lage implementatiekost waarmee verklaringen in natuurlijke taal kunnen gegeven worden voor regels die verbroken zijn. De duidelijkheid van de verklaringen blijkt sterk af te hangen van de theorie, en daarbij is de uitleg zeer statisch. Afhankelijk van het probleem kan overwogen worden om gebruik te maken van reified constraint. De laatste techniek tenslotte, gebruikt een automaat voor het berekenen van minimale oplossingen om de selectie terug satisfieerbaar te maken. De manier waarop deze automaat wordt gebouwt is z\'{e}\'{e}r interessant omdat het gebruik maakt van model expansie, een inferentie techniek die ingebouwd zit in IDP. De resultaten van deze automaat zijn heel positief en het feit dat ze kan gebouwd worden met behulp van IDP biedt mogelijkheden voor de toekomst. 


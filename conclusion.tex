\chapter{Conclusion}
\label{cha:conclusion}
\paragraph{Proof of concept}
De eerste onderzoeksvraag die gesteld werd in dit onderzoek was de vraag of het mogelijk was om met behulp van IDP een theorie te ontwikkelen voor het ISP probleem en belangrijker nog of we effici\"{e}nt inferentie kunnen doen met deze theorie. De voordelen van de uitdrukkingskracht van FO(\textperiodcentered) zijn al vaker aangehaald \cite{van2016kb} \cite{de2014predicate} in het verleden. En voor het huidige probleem houden deze beweringen opnieuw stand, sinds ik met behulp van FO(\textperiodcentered) erin geslaagd ben om een theorie te ontwikkelen waarin de regels van het ISP vervat zitten. Deze theorie is zelfs in staat om meerdere opleidingen binnen het domein van informatica en computerwetenschappen te beschrijven. 
In IDP en het concept van kennis representatie is er een strikte scheiding tussen kennis en inferentie. De theorie wordt slechts \'{e}\'{e}nmaal beschreven en deze laten toe om meerdere vragen te beantwoorden d.m.v. veschillende vormen van inferentie. Dit is enkel nuttig als deze inferentie technieken effici\"{e}nt en snel werken, en in dit geval snel genoeg om te kunnen voldoen aan de eisen van een interactief configuratie probleem. Het testen van deze technieken werd gedaan met behulp van de zelf ontwikkelde front-end applicatie. Voor alle gebruikte technieken bleken de reactiesnelheden ruim onder hun grenzen te liggen. Zelfs voor de meest complexe bewerkingen (bv. minimizatie) boekt het IDP systeem goede resultaten. Dus met dit proof of concept wordt opnieuw de kracht van FO(\textperiodcentered) getoond samen met de prestaties van de inferentiemethoden binnen het paradigma van kennis representatie.

\paragraph{Conflict explanation}
Meerdere onderzoeken hebben de voordelen van declaratieve methoden \cite{gelle1996interactive} en specifiek die van KR-systemen \cite{de2014predicate} \cite{denecker2008building} \cite{van2016kb} \cite{vlaeminck2009logical} ten opzichte van imperatieve strategie\"{e}n al aangehaald. Dit betekent niet dat een declaratieve aanpak op alle vlakken beter is, \'{e}\'{e}n voorbeeld daarvan is het domein van conflict explanation. In deze paper heb ik een aantal interessante technieken ge\"{i}mplementeerd en de resultaten ervan beschreven. Twee hiervan, de unsatstructure en reified constraints maken enkel gebruik van de tools beschikbaar in IDP en \'{e}\'{e}n methode combineert technieken binnen IDP met die van \cite{amilhastre2002consistency}. De unsatstructure werkt zoals verwacht, het is een ideale tool voor het opsporen van de oorzaak van niet-satisfieerbaarheid. Toegevoegd hieraan is de techniek van reified constraints, waarmee verklaringen in natuurlijke taal kunnen gegeven worden voor regels die verbroken zijn. Hoewel deze aanpak gelimiteerd is in de verklaringen die het kan geven is ze daarbij ook nog eens sterk afhankelijk van de theorie. Gebaseerd op de in dit onderzoek gebruikte theorie, blijkt de uitleg over het algemeen voldoende duidelijk om de gebruiker te kunnen helpen in eventuele conflict situaties. Het grootste voordeel van reified constraints is uiteraard de lage implementatiekost. De laatste techniek tenslotte, gebruikt een automaat voor het berekenen van minimale oplossingen om de selectie terug satisfieerbaar te maken. De manier waarop deze automaat wordt gebouwt is z\'{e}\'{e}r interessant omdat het gebruik maakt van model expansie, een inferentie techniek die ingebouwd zit in IDP. De resultaten van deze automaat zijn heel positief en het feit dat ze kan gebouwd worden met behulp van IDP biedt mogelijkheden voor de toekomst. 


\chapter{Toekomstig Werk}
\label{cha:toekomstigwerk}
Voor een klein domein van opleidingen is het gelukt om een theorie te ontwerpen met behulp van FO(\textperiodcentered). Natuurlijk is het aantal opleidingen binnen de K.U. Leuven aanzienlijk groter. Dus achterhalen of het mogelijk is om een theorie te ontwikkelen die om kan gaan met alle opleidingen, is werk voor de toekomst. Maar ik zou niet enkel zoeken naar aan antwoord op dit probleem maar ook naar alternatieven zoeken, bijvoorbeeld \'{e}\'{e}n theorie per type opleiding. Bachelor opleidingen hebben vaak meer gemeen met andere bachelor opleidingen, dan met bijvoorbeeld master studies. Het zou zeker interessant zijn om te kijken wat de mogelijkheden hieromtrent zijn en voor- en nadelen die hiermee gepaard gaan. 

De prestaties van de inferentie technieken van IDP voor het ISP probleem zijn positief en voldoen aan de vereisten van een IC probleem. De front-end is in staat om alle functionaliteit van de bestaande web applicatie aan te bieden met toevoeging van nieuwe functionaliteiten. Deze steunen \textit{bijna} allemaal op de technieken die in IDP te vinden zijn. Deze succesvolle steekproef op een klein subdomein van opleidingen leidt mij tot de conclusie dat er meer mogelijk is. Hopelijk zijn deze resultaten een motivatie voor anderen om verder uit te zoeken wat de mogelijkheden zijn van IDP met betrekkingen tot het ISP probleem en andere (als niet alle) opleidingen aan de K.U. Leuven en misschien zelfs andere onderwijsinstellingen. 

Door gebruik te maken van de automaat die beschreven staat in het werk van Amilhastre et al. is het gelukt om minimale correcties te kunnen berekenen. Opvallend hierbij is dat de bouw hiervan kan gebeuren op basis van model expansie in IDP. Het onderzoek rond IDP is nog altijd volop bezig, en een nieuwe versie van het systeem wordt momenteel ontwikkeld. Gezien de voordelen van het gebruik van zo'n automaat zou het zeer interessant zijn om te zien of het werk van Amilhastre niet ge\"{i}ntegreerd kan worden in een toekomstige versie van het systeem. Het werk van \citet{fargier2004compiling} is een uitbreiding op dat van Amilhastre en Vempaty. Fargier stelt het gebruik van Tree-driven automata voor, een nog compactere weergave waarbij de positieve eigenschappen van voorheen blijven gelden. En verder onderzoek zal antwoordt moeten bieden of deze techniek eventueel ook gecombineerd zou kunnen worden met het IDP systeem.
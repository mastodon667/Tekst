\chapter{Introduction}
\label{cha:intro}

\subsection{Constraint Satisfaction Problems}
CSP's zijn van het type problemen waar men voor een reeks variabelen een geldige waarde uit het domein dient toe te kennen. De toekenning echter is gebonden aan een set van regels waaraan de variabelen moeten voldoen. Formeel defini\"{e}ren we een CSP als een triple $\langle \mathcal{X},\mathcal{D},\mathcal{C} \rangle$, met $\mathcal{X}$ de verzameling van variabelen, $\mathcal{D}$ de domeinwaarden voor deze variabelen en $\mathcal{C}$ de constraints over de variabelen. Typisch worden dit soort problemen opgelost d.m.v. zoeken door alle mogelijk combinaties, deze recursieve methode gaat voor elke nieuwe toekenning na of de constraints voor de parti\"{e}le toekenning consistent zijn. Zo ja, dan volgt er een nieuwe recursieve oproep, anders gaat het algoritme backtracken. Het oplossen van CSP's met een eindig domein valt in de klasse van NP-Compleet problemen met betrekking tot de grootte van het domein. Om het zoekproces proberen te versnellen bestaan er variaties op backtracking zoals backmarking en backjumping. De eerste zorgt voor een effici\"{e}ntere manier om consistentie na te gaan en backjumping is een betere manier van backtracking waarbij men over meerdere waarden tegelijk kan backtracken. Deze technieken versnellen het zoekproces in dat ze onnodige stappen detecteren en overslaan. Anderzijds bestaan er propagatie technieken die lokale consistentie garanderen met technieken zoals arc consistency, hyper-arc consistency en path-consistency die de domeinen van de variabelen proberen te verkleinen alvorens het zoekproces gestart wordt. 


\subsection{Interactive Configuration Problems}
In deze thesis ligt de focus op een specifieke groep van problemen binnen het domein van CSP's genaamd Interactieve Configuratie (IC) problemen. Constraint Satisfaction Problems zijn problemen waar men een toekenning van domeinwaarden voor de variabelen wil zoeken waarvoor geldt dat aan alle constraints voldaan is. Bij IC problemen wil men exact hetzelfde bekomen maar waar men voorheen recursief ging zoeken naar zo'n oplossing is het nu een gebruiker die handmatig stap per stap een geldige toekenning gaat proberen selecteren. Om de gebruiker hierin bij te staan is er nood aan software, software die de gebruiker zo goed mogelijk bijstaat gedurende dit hele proces. Men weet uit ervaring dat het schrijven van software voor dit soort problemen geen gemakkelijke opgave is. Regels ontwikkelen voor kleine problemen zoals het n-queen probleem zijn intu\"{i}tief en gemakkelijk. Maar naarmate de problemen groter en complexer worden zal het als maar moeilijker worden om dit in software om te zetten. Verder is aangetoond dat een imperatieve aanpak voor het beschrijven van de regels van een probleem vaak moeilijk is \citep{gelle1996interactive}. Reden hiervoor is dat de regels over de betreffende domeinkennis vaak verspreid zit in de software in codesnippets. Daarboven is onderhoud van zulke software een enorm moeilijke opgave waarbij de kleinste wijziging in de constraints kan leiden tot een volledige herwerking van de code. Daarom is men op zoek gegaan naar alternatieven en het antwoord kwam uit de declaratieve hoek. Met de opkomst van declaratieve programmeertalen, vooruitgang in automated reasoning en als maar toenemende rekenkracht van computersystemen is de populariteit van declaratieve toepassingen alleen maar gegroeid. In declaratieve toepassingen kan men de regels van een probleem beschrijven d.m.v. logische constraints. Wellicht het grootste voordeel van declaratieve methoden in tegenstelling tot imperatieve is dat de regels overzichtelijker en duidelijker zijn. Onderhoud van deze constraints is veel makkelijker, als er een regel wijzigt dan hoeft ook enkel deze constraint aangepast te worden. En hoewel men met een imperatieve aanpak optimalere algoritmen kan ontwikkelen, heeft men kunnen laten zien dat declaratieve methoden ook goede prestaties kunnen neerzetten \citep{vlaeminck2009logical}. 

\subsection{Knowledge Representation: Het IDP Systeem}
De interesse in declaratieve talen is de laatste decennia toegenomen en als resultaat zijn er een hele resem talen verschenen (Prolog, Ant, Lisp, ...). Wat opvalt is dat elke taal vaak samenhangt met \'{e}\'{e}n enkele specifieke vorm van inferentie. En toch is ondanks de verschillende vormen van inferentie de kennis over het domein hetzelfde. Enkele jaren geleden heeft men het concept van Knowledge Representation \citep{denecker2008building} voorgesteld. In het KR-paradigma stelt men dat ongeacht de inferentie die men wil toepassen, men hiervoor telkens dezelfde kennis kan hergebruiken. Deze kennis is niets meer dan een verzameling van informatie, maar met deze informatie kan men meerdere vormen van inferentie toepassen. /*UITLEG OVER IDP ZELF*/

\section{Probleemstelling}

\subsection{Individueel Studieprogramma}
Een ISP samentstellen valt onder deze categorie van Interactieve Configuratieproblemen. De gebruiker, in dit geval een toekomstige student wil een geldig ISP bekomen d.m.v. het selecteren van mogelijke opleidingsonderdelen. Maar de student kan niet zomaar elk opleidingsonderdeel selecteren, er zijn een heleboel regels waaraan de selectie moet voldoen. Deze regels zijn op zichzelf duidelijk en intu\"{i}tief, maar om een selectie vinden die aan alle regels voldoet kan mogelijk verwarrend zijn. Het huidige systeem voorziet weinig ondersteuning in het bijstaan van de gebruiker gedurende het selectieproces. Het is pas als je een selectie bevestigt dat het systeem controleert of deze ook effectief correct is volgens de regels. En hoewel het systeem wel weergeeft aan welke regels er (in geval van inconsistentie) niet voldaan is, moet de gebruiker zelf op zoek gaan naar de selectie die verantwoordelijk is verantwoordelijk is hiervoor. De regels van een geldig ISP nemen geen lessenrooster in acht, wat dus betekent dat lesmomenten voor verschillende opleidingsonderdelen kunnen overlappen. 

\section{Doel}
In deze thesis willen we onderzoeken of de regels van het ISP effici\"{e}nt kunnen beschreven worden in FO(\textperiodcentered). De regels zijn voor elke opleiding anders. Daarom wil ik graag onderzoeken of het mogelijk is om met \'{e}\'{e}n enkele theorie in IDP alle opleidingen correct te kunnen beschrijven. Belangrijk hierbij is dat de domeinen sterk kunnen verschillen tussen opleidingen, en dat de eventuele theorie hier mee om moet kunnen gaan.

Een minpunt van het huidige systeem is de minimale ondersteuning die het biedt tijdens het selectieproces. Hier kan beter gedaan worden, daarom wil ik meer ondersteuning proberen bieden. In een zelf ontworpen Front-end met grafische user interface wil ik de volgende functionaliteiten integreren:
\begin{description}
\item[Automatisch invullen van gevolgen] Als de student een vak A kiest en hieruit volgt dat vak B ook gevolgt moet worden, dan is het de bedoeling dat het systeem dit automatisch invult zodat de student zich hier verder niets van hoeft aan te trekken. 
\item[Detectie van foutieve selectie] Deze functionaliteit is momenteel al aanwezig in het huidige systeem. Maar de detectie gebeurt pas bij de bevestiging van de selectie i.p.v. op het moment van de selectie zelf. 
/*ALLE FUNCTIONALITEITEN BESCHRIJVEN*/
\end{description}


Momenteel wordt het lessenrooster niet in acht genomen tijdens het selectieproces. Dit wil zeggen dat de student dus ook geen idee heeft of de door hem/haar geselecteerde opleidingsonderdelen mogelijk lesmomenten bevatten die overlappen. Hier is het de bedoeling om de gebruiker een overzicht te geven van het lessenrooster voor elke stap in het selectieproces. En daarboven het beste lessenrooster te laten genereren waarbij er zo weinig mogelijk overlappende lessen zijn.

In het geval van een ongeldige selectie biedt IDP een aantal mogelijkheden om aan de gebruiker duidelijk te maken wat er juist mis is gegaan. De \emph{unsatcore} zoekt naar de minimale set van regels die ongeldig zijn voor de huidige selectie. Een andere optie is de \emph{unsatstructure} die de kleinste set van variabelen wordt gezocht waarbij als men de geselecteerde waarden voor deze variabelen ongedaan maakt, de selectie niet langer ongeldig is. Maar om een duidelijk verklaring te verkrijgen waarom de selectie fout is, in een formaat dat elke persoon kan begrijpen bestaat er in IDP momenteel nog niets. Daarom willen we onderzoeken of we dit kunnen realiseren.

\section{Dependencies}
\begin{description}
\item [FO(\textperiodcentered)] Het is de bedoeling om de mogelijkheden van IDP in actie te zien en met een proof of concept de prestaties van het systeem te testen. FO(\textperiodcentered) bezit een grote uitdrukkingskracht, bovenop eerste order logica bevat het ook taalelementen zoals aggregaten (sum, avg, min, max, ..) en inductieve definities. 
\item [Kivy] Bij de keuze van de programmeertaal en framework viel mijn oog op Kivy. Dit is een Python library met een hele waaier aan grafische elementen die elk uitgebreid geconfigureert kunnen worden. Hiervoor kan gebruik gemaakt worden van een door de uitgever ontwikkelde kv-language. Kivy applicaties werken cross-platform en de applicaties zijn event-driven gebruik makend van een centrale loop.
\item [JSON] Het gegevensformaat van JSON is ideaal om de domeinen van verschillende opleidingen in weer te geven. De domeinwaarden zitten gekoppeld aan een attribuut, dit maakt het gemakkelijk om structuur te parsen en zo het domein voor iederen opleiding in te lezen.
\end{description}